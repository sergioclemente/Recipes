\documentclass{article}

\usepackage{listings}

\usepackage[a4paper, margin=0.8in]{geometry}

\title{Recipes}
\author{Sergio Clemente Filho}
\date{\today}

\begin{document}

\maketitle

\newpage

\tableofcontents{}

\newpage

\section{Flavor Profiles}
\section{Ratios}

\newcommand{\chick}[3]{
\subsection{#1}

\paragraph{Ingredients:}

\begin{itemize}
\item #2
\item 1 Chicken Breast 
\item oil
\end{itemize}

\paragraph{Directions:}
\begin{itemize}
\item Cut the chicken, wash and add some lime. An alternative way of prepping is just washing the chicken and patting with paper towels for removing excess moisture.
\item If you let it lime, let it sit for 10min, than wash the chicken with running water
\item (Optional) If you can put all ingredients in a ziplock bag and let it rest for couple of hours.
\item On a sautee pan, add some oil and sear one side of the chicken in high heat.
\item Throw in the remaining ingredients. Lower the heat, flip the side of the chicken, add a lid and let it simmer for 10-15min.
\end{itemize}

\paragraph{Notes:} #3
}

\section{Fast everyday recipes}

\chick{Curry Chicken}{Knoor, Curry}{If you can use coconut oil the chicken will taste a lot better.}

\chick{Garam Masala Chicken}{3 tsp garam masala, 2 table spoon of chopped garlic}{}

\chick{Chicken on Tomato Sauce}{1 can of diced tomato (fire roasted), cummin, cilantro, 1 knoor, chopped garlid}

\chick{Tajin chicken}{tajin seasoning}

\chick{Onion chicken}{2 onions, 2 tsp of chopped garlic, 2 tablespoons of olive oil, 1 knoor}{User a blender to blend onion and garlic. Use just enough water so you can get a pasty texture}

\chick{Curried Tomato}{2 onions, 5 chopped tomatoes, 1 tablespoon of garlic, 1 tsp cumin, 2/3 tsp paprika, 2/3 tsp curry, 1 knoor}{User a blender to blend onion and garlic.}

\chick
{Curry Chicken}{1 tsp of knoor, 1 tsp of curry, olive oil}{Use coconut oil for extra flavor}

\chick{Paprika Chicken}{Sweet Paprika (A lot - 2 tablespoons for 1 chicken breast)}

\chick{Fish sauce}{2 tbl spoon sugar, 2 garlic cloves (finely minced), fish sauce (1/4 cup)}

\chick{KitchenSync Sauce}{2 tsp sugar, 1tsp salt, 1 tbl spoon garlic (or 2 minced garlic), 1 tbl spoon balsamic vinegar, 4 tbl spoon olive oil}

\chick{Orange and Soy}{2 tbl spoon olive oil, 1/2 cup of orange juice, 1/4 cup of soy sauce and 2 garlic cloves (pureed)}

\subsection{Carne Assada}

\paragraph{Ingredients:}

\begin{itemize}
	\item 1 cut of flank steak
	\item 1 lemon
	\item 4 garlic cloves
	\item 1/2 cup of soy sauce
	\item 1 teaspoon chilli powder
	\item 1 teaspoon paprika
	\item 1 teaspoon ground cumin
	\item 1 teaspoon oregano
	\item black pepper
	\item olive oil
\end{itemize}

\paragraph{Directions:}
\begin{itemize}
	\item Mix all ingredients
	\item Let it sit in the fridge overnight
	\item On a high heat, put meat first
	\item Then add the juices and sauce
	\item Simmer for 10-15'
\end{itemize}

\subsection{Soy Meat}

\paragraph{Ingredients:}

\begin{itemize}
	\item beef for beef stew (1lb)
	\item 1 onion
	\item 2-3 garlic cloves (1 table spoon of chopped garlic)
	\item 1/4 red pepper (Optional)
	\item 1/4 tea spoon ginger powder (Optional)
	\item 1 cup soysauce (low sodium)
\end{itemize}

\paragraph{Directions:}
\begin{itemize}
	\item Add a bit of oil to a saute pan, add onion and red pepper for 2-3min, then add garlic and pepper. Then add the garlic until its fragrant.
	\item Reserve the onion and pepper.
	\item On the same pan add the beef, sear the beef all sides
	\item When beef is seared (4-5min) add the soy sauce, optionally the ginger and let it cook for 10-15min.
\end{itemize}

\subsection{Chicken (Receita Casa Vovo)}

\paragraph{Ingredients:}

\begin{itemize}
	\item 1 cebola 
	\item 1 tomate 
	\item 1/2 pimentao 
	\item cebolinha (o verde e branco tudo junto) 
	\item 2-3 dentes de alho 
	\item Coentro 
	\item 2 caldos quinor (pra 2 peito de frango) 
	\item Molho de tomate 
	\item Vinagre branco
\end{itemize}

\paragraph{Directions:}
\begin{itemize}
	\item Refoga com azeite os ingredients.
	\item Corta a galinha em pedacos e coloca na agua e limao. Depois de lavar a galinha,
	\item Joga a galinha no refogado (Nao refoga mto nao, so ate o knor derreter). 
	\item Joga o cuminho com sal.
	\item Bota molho de tomate e vinagre (1 colher)
	\item Deixa por uns 30-40min em fogo baixo tampado.
\end{itemize}

\section{Starches / Beans}

\subsection{Refried beans}

\paragraph{Ingredients:}

\begin{itemize}
	\item Already made beans (2 cups)
	\item Mexican Spice Blend (2 tsp paprika, 2 tsp chili powder, 1/2 tsp garlic salt, 1/2 tsp onion salt, 1 tsp cumin, pepper to taste)
\end{itemize}

\paragraph{Directions:}
\begin{itemize}
	\item Put beans on a blender and blend until forms a puree
	\item On a sauce pan add mexican spice blend.
	\item Reduce heat, let it reduce. 5'.
\end{itemize} 

\subsection{Rice}

\paragraph{Ingredients:}

\begin{itemize}
	\item 1 cup of rice
	\item 2 cups of water
	\item 1 table spoon of knoor
\end{itemize}

\paragraph{Directions:}
\begin{itemize}
	\item Boil water, then bring it to a simmer
	\item Add rice and knoor
	\item Cook covered for 20-30'
	\item Turn heat off (don't uncover) and let it sit for 20'
\end{itemize} 

\subsection{Feijao de Soja Luzinete}

\paragraph{Ingredients:}

\begin{itemize}
	\item 2 cups of soybeans
	\item olive oil
	\item 1 onion
	\item 1 tomato
	\item Tomato sauce (The brazilian one Tarantella is the best)
\end{itemize}

\paragraph{Directions:}
\begin{itemize}
	\item Cook the soybeans on the pressure cooker with just water and salt for 40'.
	\item On a sautee pan cook chopped onions and tomato
	\item Add the tomato sauce and cooked soybeans
	\item Cook for another 10'
\end{itemize}

\subsection{Potatoes Gratin (Sergio)}

\paragraph{Ingredients:}

\begin{itemize}
	\item Parmesan
	\item Whipping Cream
	\item Cream Cheese Whipped
	\item Mayo
	\item Oregano
	\item Musarella Cheese
	\item Potatoes
\end{itemize}

\paragraph{Directions:}
\begin{itemize}
	\item Cut the potatoes in small cubes and boil until is al dente.
	\item Mix all the other ingredients, mix with the potatoes. Top with musarrela cheese and put in the oven on 450F for 30min.
\end{itemize}

\subsection{"Risoto" Brasileiro}

\paragraph{Ingredients:}

\begin{itemize}
	\item Arroz
	\item 2 colheres de trigo
	\item 1 xicara de leite
	\item Queijo parmesao
	\item Manteiga
\end{itemize}

\paragraph{Directions:}
\begin{itemize}
	\item Cozinhe o arroz normal
	\item Bata no liquidificador o leite e o trigo
	\item Quando o arroz tiver pronto na mesma panela do arroz coloque o leite com farinha
	\item Coloque 1 tbl spoon de manteiga, parmesao e sal a gosto
	\item Misture ate engrossar
\end{itemize} 

\subsection{Feijoada}

\paragraph{Ingredients:}

\begin{itemize}
	\item 0.5 Kg of Black beans 
	\item 300g de charque (any salted meat will do) 
	\item 1 Calabresa Sausage (pork, salt, etc) - smoked turkey sausage 
	\item 1 piece of pork ribs 
	\item Coloral (similar to paprika but spicier) 
	\item Bay leaves 
	\item Tomatoe sauce 
	\item Cilantro 
	\item 1 onion 
	\item 1 tomatoe 
	\item 1/4 green pepper 
	\item vinegar
\end{itemize}

\paragraph{Directions Ribs:}
\begin{itemize}
	\item Sautee with Cumin, Paprika, garlic and vinegar (a bit) and salt
\end{itemize}

\paragraph{Directions Salted beef (Charque):}
\begin{itemize}
	\item Boil with water to remove excess of salt. - or throw some hot water into it. 
\end{itemize}

\paragraph{Directions Beans:}
\begin{itemize}
	\item Put beans in water for 1-2 hrs
	\item Mix beans, water, ribs, beef and bay leaves and cook under pressure for 15-25min.
	\item Take some beans with water, blend with: a. Add cilantro, 1/2 tomato, 1/2 onion, 1/4 green pepper chopped, tomato paste (You can also blend this in the beginning and let it cook on pressure with the beans altogether)
	\item Pour back
	\item Add sausage and boil for about 10min 
	\item Add salt if necessary
\end{itemize}

\paragraph{Notes:}
Easy mode of this dish is put everything in the slow cooker (high mode) for 5hrs.

\section{Indian food}

\subsection{Sag Paneer (With tofu)}

\paragraph{Ingredients (For gravy):}
\begin{itemize}
\item 2 bunches of spinach (500g)
\item 1 pinch of kasuri methi
\item 1 tablespoon of tomato paste
\end{itemize}

\paragraph{Ingredients (Base):}
\begin{itemize}
\item 1 pinch of cumin seeds
\item 1 onion thiny sliced
\item 1 teaspoon of grated fresh ginger
\item 3 cloves garlic, minced
\item 1 tomatoe diced
\end{itemize}

\paragraph{Ingredients (Seasonings):}
\begin{itemize}
\item 2 tsp of garam masala
\item 1/8 tsp of cardamom
\item 1/2 tsp ground turmeric
\item 1/2 tsp red pepper
\item 1 tsp cumin
\end{itemize}

\paragraph{Ingredients (Protein):}
\begin{itemize}
\item Tofu
\end{itemize}

\paragraph{Directions:}
\begin{itemize}
\item Dry tofu with paper towels. Fry them and set aside.
\item In a pan add oil and fry cumin seeds (Around 1 minute). \item Add onion for 2-3 minutes. Then add ginger and garlic until fragrant. Add seasonings (garam masala, turmeric, cumin, red pepper and cumin). Add the tomato and sear for 10'.

\item Add in the spinach, kasuri methi, tomato paste and simmer for 5' (You can also cook the spinach and make a puree before). Add the friend tofu. 
\end{itemize}|

\section{Breakfast}

\subsection{Waffle}

\paragraph{Ingredients:}

\begin{itemize}
	\item 2 eggs
	\item 2 cups of all purpose flour
	\item 2 cups of milk (or nido - 1/2 cup powder + 2 cups of water)
	\item 1/2 cup of vegetable oil
	\item 1 tbl spoon sugar
	\item 4 tsp baking powder
	\item 1/4 tbl spoon salt
	\item [Optional] Mozzarella cheese and turkey (chopped)
\end{itemize}

Have some spare butter

\paragraph{Directions:}
\begin{itemize}
	\item Preheat waffle iron. Beat eggs in large bowl with hand beater until fluffy. Beat in flour, milk, vegetable oil, sugar, baking powder, salt and vanilla, just until smooth.
	\item Spray preheated waffle iron with non-stick cooking spray. Pour mix onto hot waffle iron (You can optionally interleave with the turkey and cheese). Cook until golden brown. Serve hot.
\end{itemize}

Experiment1: Add more salt (1/2). Add parmesan cheese.

\subsection{Pancakes}

\paragraph{Ingredients:}

\begin{itemize}
	\item 2 cups of all purpose flour 
	\item 2 teaspoon of baking powder 
	\item 1/2 teaspoon salt 
	\item 1 teaspoon sugar 
	\item 2 eggs 
	\item 1 1/2 cups of milk (Might need more milk)
	\item 2 tablespoons of melted and cooled butter
\end{itemize}

Have some spare butter

\paragraph{Directions:}
\begin{itemize}
	\item Mix all ingredients until there are no lumps 
	\item Turn heat on, let it get war before putting the butter 
	\item Use the 1/3 cup measurer to put pancakes into the pan 
	\item Flip after 3-5min in the pan, adjust heat accordingly if too hot
\end{itemize}

\section{Apps}

\subsection{Beet Humus}

\paragraph{Ingredients:}

\begin{itemize}
	\item 1 can of whole beets
	\item 1 can of garbanzo beans
	\item 1 lemon (zest + juice)
	\item 6 garlic cloves
	\item 1/4 cup olive oil
\end{itemize}

\paragraph{Directions:}

\begin{itemize}
	\item Put everything in a mixer (Add some salt and pepper to taste) and mix. You might want to put some water too.
\end{itemize}

\subsection{Bruschetta}

\paragraph{Ingredients:}

\begin{itemize}
	\item 2 riped tomatoes
	\item 6 garlic cloves
	\item basil
	\item olive oil
\end{itemize}

\paragraph{Directions:}
\begin{itemize}
	\item Heat olive oil and garlic on a skillet, when its ready put in on the side and let it rest for 20min
	\item Cut the tomatoes and basil in small pieces and mix with the garlic
	\item Season with salt and pepper
	\item Put butter on both sides of the toast, put on the oven for 350F until on of the sides is brown, then put the mixture
\end{itemize}

PS: Trick to cut the toast thin is to put on the freezer overnight


\subsection{Eggplant (Sergio)}

\paragraph{Ingredients:}

\begin{itemize}
	\item 1 eggplant
	\item 5 garlic cloves
	\item red bell peppers
\end{itemize}

\paragraph{Directions:}
\begin{itemize}
	\item Chop everything, put some olive oil and salt then put in a tray at 400F for 20-30min.
	\item Put everything in a blender with a bit of water 
\end{itemize}

\subsection{Eggplant (Cunhado)}

\paragraph{Ingredients:}

\begin{itemize}
	\item 1 Eggplant
	\item 8 garlic cloves
	\item 1 onion
	\item olive oil
	\item red pepper
	\item 1 tomato
\end{itemize}

\paragraph{Directions:}
\begin{itemize}
	\item Add olive oil in the sauce pan and cook garlic, onion and pepper
	\item Add eggplants cut into small pieces
	\item Cover sauce pan for 10'
	\item Leave in the fridge overnight for absorbing flavour
\end{itemize}

\subsection{Garlic bread}

\paragraph{Ingredients:}

\begin{itemize}
	\item 2 cups - Cream cheese (Whipped is better)
	\item 1/2 cup - Mayonese 
	\item Garlic (5 tablespoons)
	\item Parmesan
	\item Sharp Sheddar
	\item Pepper
	\item Oregano
	\item Bread (Take and bake are better)
\end{itemize}

\paragraph{Directions:}
\begin{itemize}
	\item Mix all ingredients
	\item Put oven on 425F for 10-15'
\end{itemize}

\subsection{Paprika Toasts}

\paragraph{Ingredients:}

\begin{itemize}
	\item Smoked Paprika
	\item Olive oil
	\item Cream cheese
	\item Cheese (Mozzarella, Parmesian)
	\item Salt
	\item Toasts
\end{itemize}

\paragraph{Directions:}
\begin{itemize}
	\item Mix all ingredients (You can also use mayo if you want)
	\item Turn oven on 350F and once hot add toasts before so that its slightly toasted before you add the spread - 5' each side
	\item Remove from oven put spread
	\item Put for more 5' in the oven
\end{itemize}

\section{Weekend Meals}

\subsection{Tomato Soup}

\paragraph{Ingredients:}
\begin{itemize}
	\item 2 lb of tomatoes
	\item 7 garlic cloves
	\item 1 onion
	\item Chicken stock
	\item olive oil
	\item 3 table spoons Parmesan cheese
\end{itemize}

\paragraph{Directions:}
\begin{itemize}
	\item Cut tomatoes in half.
	\item Roughly chop onions
	\item Put all vegetables in a large tray.
	\item Season generously with salt and pepper
	\item Sprinkle olive oil on top
	\item Put tray on oven on 375F for 60-90minutes
	\item When vegetables are roasted put in a food mixer
	\item Put mixture in a sauce pan, add chicken stock until reaches desired consistency.
	\item Taste and depending on acidity add some sugar (1-2 tsp of coconut sugar). Make sure to balance with salt for sweetness.
\end{itemize}

\paragraph{Notes:}
\begin{itemize}
	\item (Optional) Add 1 tsp of cornstarch and 1 cup of milk for a richer soup.
	\item (Optional) Add 2-4 tbl spoon of Parmesan cheese.
\end{itemize}

\subsection{French Onion Soup}

\paragraph{Ingredients:}

\begin{itemize}
	\item 2 tbl spoons unsalted butter
	\item 3 large onions (2 pounds), halved lengthwise and thinly sliced crosswise
	\item sea salt
	\item fresh ground pepper
	\item 2 tablespoons dry sherry
	\item 1 quart rich beef stock
	\item 1 bouquet garni, made with 1 bay leaf, 1 thyme sprig, 2 juniper berries and 2 flat-leaf parsley sprigs, tied in cheesecloth
	\item 2 cups shredded Gruyère cheese (about 6 ounces)
\end{itemize}

\paragraph{Directions:}
\begin{itemize}
	\item Melt the butter in a large enameled cast-iron casserole. Add the onions and a pinch of salt, cover and cook over moderate heat, stirring once or twice, until the onions soften, about 10 minutes. Uncover and cook over moderate heat, stirring frequently, until the onions are lightly browned, about 40 minutes.
	\item Stir in the sherry. Add the stock and bouquet garni and bring to a boil. Cover and simmer over low heat until the soup has a deep flavor, about 30 minutes. Discard the bouquet garni and season the soup with salt and pepper.
	\item Preheat the oven to 425°. Bring the soup to a simmer, ladle it into 4 deep ovenproof bowls and sprinkle with half of the cheese. Bake the bowls of soup on a baking sheet in the middle of the oven for 10 minutes, or until the cheese is bubbling. Serve hot.
\end{itemize}

\subsection{Hamburgers}

\paragraph{Ingredients:}

\begin{itemize}
	\item Good quality ground beef
	\item 1/2 onion
	\item 2-3 cloves of garlic
	\item [garlic salt and onion salt] or
	\item [1 table spoon of finaly chopped bacon]
	\item salt
	\item pepper
	\item olive oil
	\item cheese
\end{itemize}

\paragraph{Directions:}
\begin{itemize}
	\item On a sautee pan add onion for 2-3' then garlic
	\item When onion is caramelized, turn off the heat. Reserve.
	\item On a bowl, mix ground beef, [onion salt, garlic salt] or [bacon], salt and pepper
	\item Make the patties, using the same pan add 1 tablespoon of olive oil and 1/2 part of butter. Sear both sides of steak than cover so it can cook.
	\item While burger is cooking, but buns into the oven at 400 for 5min.
	\item Wait for 5', add cheese on top, cover again for 1'
\end{itemize} 

\subsection{Boef Bourguignon}

\paragraph{Ingredients:}

\begin{itemize}
	\item 4 ounces of bacon, sliced crosswise into thin strips (1/4 inch by 1 inch pieces)
	\item 2 pounds of boneless beef chuck, trimmed of all fat and cut into 1 inch cubes
	\item 1 carrot, peeled and sliced into 1/4 inch think rounds
	\item 1 large yellow onion, medium diced
	\item 2 tablespoon unbleached all-purpose flour
	\item 2 cups full-bodied, red wine like chianti or merlot
	\item 2-3 cups of low sodium beef stock
	\item 1 tbl spoon tomato paste
	\item 3 large garlic cloves, finely minced
	\item 3 sprigs fresh thyme, tied with butcher's twine
	\item 2 bay leaves
	\item 15 pearl onions
	\item 2 tbl spoon of butter, room temperature
	\item 8 ounces cremini or button mushrooms, trimmed and quartered
	\item 1/3 cup sherry vinegar
	\item 3 tbl spoons chopped flat-leaf parsley
\end{itemize}

\paragraph{Main:}
\begin{itemize}
\item Preheat oven to 325
\item In a large dutch oven over medium heat, saute bacon until lightly browned. About 5 minutes. Transfer bacon into a medium bowl and reserve fat in the pan.
\item Dry the beef thoroughly with paper towel and season with salt and pepper. Return dutch oven to medium-high heat. When fat is shimmering, add the beef and sear all sides. Transfer seared beef to bowl with bacon.
\item Reduce heat to medium and add the carrots and onions to the pan. Saute until lightly browned, about 8 minutes.
\item Return the beef and bacon to the dutch oven with the onion and carrots; season lightly with salt and pepper. Sprinkle flour over and toss to a lightly coat.
\item Slowly stir in wine and add enough stock to just cover the meat.
\item Stir in tomato paste, garlic, thyme sprigs and bay leaf.
\item Bring to a simmer over medium heat. 
\item Cover the pot and place in the preheated oven for 1 1/2 to 2 hours.
\item The meat is done when a fork pierces it easily.
\item Separate the solids from the liquid with a sieve. \item Discard thyme sprigs and bay leaf.
\item Put liquid in a saucepan with the sherry vinegar and let is simmer until liquid is required by half.
\item Add roasted onions and mushroom to the dutch oven.
\item Skim fat off the surface of the braising liquid with a laddle and discard fat.
\item Season with salt and pepper. Maybe adding a bit of sugar to balance the acidity (Used 3 teaspoons of coconut sugar).
\item Pour mixture to dutch oven and stir to combine.
\end{itemize}

\paragraph{Pearl onions and Mushroom}
\begin{itemize}
\item While the beef is cooking, blanch the pearl onions in rapidly boiling salted water until tender, about 4 minutes. \item Drain and let cool, then trim ends and peel. Set aside
\item To a large ovenproof skillet over medium-high heat, add butter.
\item Add onions and mushrooms and stir to coat vegetables with the melted butter.
\item Drizzle with sherry vinegar and season generously with salt and pepper.
\item Place skillet in a oven, stirrring occasionally, until onions and mushrooms turn golden brown and most of the liquid have evaporated. Around 20-25minutes.
\item Remove from oven and set aside.
\end{itemize}

\subsection{Beef Ragu (Tia Angela)}

\paragraph{Ingredients:}

\begin{itemize}
	\item 1 good cut of meat (I actually used last time Skirt Steak)
	\item 1 Bottle of 12oz of Guinness Beer
	\item Olive oil
	\item 1 jar of tomato sauce (Used Barilla tradition)
	\item 1 Packet of Onion Soup (Used lipton)
	\item 1 tablespoon of sugar
\end{itemize}

\paragraph{Directions:}
\begin{itemize}
	\item In the pressure cooker add olive oil and tablespoon of sugar in the pan (Original recipe uses teaspoon of sugar)
	\item Sear all sides of the steak
	\item Throw remainder of ingredients (Tomato sauce, Beer and Onion Soup), stir until mixture is uniform.
	\item Leave in the pressure cooker for 35-40'
\end{itemize}

\paragraph{Note: The chicken version of this dish you replace beer for 6oz of white wine (half those small bottles), [chicken stock] and put grated onion instead of the onion soup}

\subsection{Bacalhau Gomes Sa (Tia Ana)}

\paragraph{Ingredients:}

\begin{itemize}
	\item Bacalhau (1.5kg)
	\item Batata (6 batatas)
	\item Cebola (6 cebolas)
	\item Alho (1 cabeca de alho)
	\item Azeite (2 latas de azeite galo)
	\item Tomate (4 tomates)
	\item Ovo (8 ovos)
	\item Salsa (1 bocadinho)
\end{itemize}

\paragraph{Directions:}
\begin{itemize}
	\item Doura cebola e alho no azeite (Bastante azeite). Nao colocar no oleo quente. Deixar a cebola e o alho cozinhar no azeite.
	\item Cozinha o bacalhau na agua e sal por pouco tempo, separa e desfia (Se o bacalhau ja for salgado não precisa colocar mais sal
	\item Cozinha batata cortada em cubinhos. Nao deixar cozinhar muito pra não virar um "pure" quando misturar no bacalhau depois
	\item Quando a cebola comecar a dourar, joga o peixe e deixa uns 10min
	\item Coloca salsa picada
	\item Coloca a batata e espera ver se vai precisar botar mais azeite
	\item Coloca ovo cozido amassado no bacalhau (so coloca 4) os outros voce coloca fatiado por cima
	\item Deixa descansar durante a noite (pode botar em tupperware desde que esteja frio).
	\item No dia seguinte corta umas rodelas de ovo cozido e poe por cima e umas fatias de tomate tambem. Bota no forno pra deixar  queimado em cima.
\end{itemize}

\subsection{Steakhouse Steak}

\paragraph{Ingredients:}

\begin{itemize}
	\item 2 good steaks (Ribeye or New york - look for dry aged prime beef)
	\item 5 cloves of garlic
\end{itemize}

\paragraph{Directions:}
\begin{itemize}
	\item Turn oven at 450F
	\item Pat steak with a paper towel to remove any excess of moisture
	\item Season with kosher salt, group pepper and maybe coriander also.
	\item Let it sit for 10'
	\item Heat skillet
	\item Just before adding the steak add butter and a touch of olive oil.
	\item (If you bought a cheap steak you might want to add garlic)
	\item Sear each side of the steak (~2' each)
	\item Put the steak on the oven for 10-15' (For medium)
	\item Let it sit for 2-3' before serving.
\end{itemize}

\subsection{Engorda Marido}

\paragraph{Ingredients:}

\begin{itemize}
	\item Ground Beef Recipe
	\item Mashed Potatoes Recipe
	\item Mosarella cheese
\end{itemize}

\paragraph{Directions:}
\begin{itemize}
	\item Turn oven to 425F
	\item In a container put the ground beef and top with the mashed potatoes.
	\item Top over with cheese
	\item Wait for 20-30min until cheese melted and formed a crust
\end{itemize}

\subsection{Strogonoff}

\paragraph{Ingredients:}

\begin{itemize}
	\item 1 grated onion
	\item 800g of meat cut into cubes
	\item 4 tbl spoons of ketchup
	\item 1 tbl spoon of Worcestershire sauce
	\item 1 tbl spoon of mustard (the yellow cheap one)
	\item 2 tbl spoon of oil
	\item 1/2 cup of whipped cream
	\item 1 can of mushrooms
\end{itemize}

\paragraph{Directions:}
\begin{itemize}
	\item Season the meat with salt and pepper. Let it sit for 10-15'
	\item In the high heat, add onion, then throw the meat. 
	\item Once the meat is seared, lower the heat, add ketchup, mustard, Worcestershire sauce and mushrooms.
	\item Let it simmer for 10'
	\item Add whipped cream and simmer for 3-5'. Be careful not to boil.
\end{itemize}

\paragraph{Notes:}
\begin{itemize}
	\item You can make a chipotle strogonoff by using 1 table spoon of chipotle pepper (Best brand is Embasa: Chipotle peppers in adobo sauce) instead of the ketchup, worcestershire sauce and mustard.
	\item You can replace the whipping cream with 1 1/2 cup of milk and 2 tsp of cornstarch. But you have to thicken the milk by mixing on the heat for a few minutes.
\end{itemize}

\subsection{Torta de Ana}

\paragraph{Ingredients (For dough)}

\begin{itemize}
	\item 6 cups of flour
	\item 2 egg yolks
	\item 2 table spoons of Parmigiano cheese
	\item 7 tbl spoon of mayonnaise
	\item 3 blocks of butter - or 1 2/3 cup of margarine
\end{itemize}

\paragraph{Ingredients (For chicken)}

\begin{itemize}
	\item 2 chicken breasts
	\item lime
	\item creme de leche
	\item 1 onion
	\item 1 tomato
	\item 1/2 bell pepper
	\item scallion
	\item 3 garlic cloves
	\item 2 knoors (one for each breast)
	\item cumin
	\item cilantro
	\item tomato paste
	\item olive
	\item green peas
\end{itemize}

\paragraph{Directions:}
\begin{itemize}
	\item Cut chicken in small cubes put in a container with water and lime. Let it sit for a couple of minutes.
	\item Sear the vegetables with the knoor (onion, tomatoes, peppers, scalion and garlic).
	\item After the garlic is fragrant add the chicken, cumin, cilantro and tomatoe paste. 
	\item Let it simmer for 30'
	\item While chicken cooks, make the dough (mix all ingredients).
	\item When the chicken is ready, shred the chicken and put the juices on the side.
	\item Mix chicken, creme de leche and sauce to taste. 
	\item Add olives and green peas
	\item Wait for chicken to cool
	\item Put dough in a pie container (use the plastic wrap technique for building the top)
	\item Put chicken and cover the top
	\item Use egg yolks and parmesan to decorate the top
	\item Turn oven to 325F and put pie for 30'
	\item Increase heat to 425F for finish
\end{itemize}

Note:

\begin{itemize}
\item You can use the dough if any left to make some snacks. Add some smoked paprika with Parmigiano cheese.
\item If any sauce from the chicken is left, its an excellent base for rice.
\end{itemize}


 

\subsection{Bife de Molho Luzinete}

\paragraph{Ingredients:}

\begin{itemize}
	\item Beef cut into thin steaks
	\item 1 onion
	\item 1 tomato
	\item bunch of cilantro
	\item cumin
	\item 2 garlic cloves
	\item 1 knoor block
	\item tomato sauce (1 1/2 cup)
	\item 1/2 cup of water
	\item 1 cup of mozzarella cheese
\end{itemize}

\paragraph{Directions:}
\begin{itemize}
	\item Season the beef with knorr, cumin and garlic
	\item On a sauté pan sear the beef both sides
	\item Add the (onion, tomato, cilantro, tomato sauce and water)
	\item Put in the pressure cook for 15'
	\item If require cook without pressure for more 10'
	\item Turn the oven on 425F
	\item Put beef on a bowl, add mozzarella cheese and put on the oven for 15' or until cheese is melt.
\end{itemize}

\subsection{Breaded chicken}

\paragraph{Ingredients:}

\begin{itemize}
	\item Panko breadcrumbs
	\item Good Seasons Salad Dressing Recipe Mix, Italian
	\item Egg whites
	\item Chicken Breast
	\item Lime
	\item Parchment paper
\end{itemize}

\paragraph{Directions:}
\begin{itemize}
	\item Cut the chicken breast in half, use a “hammer” to flatten it out and put lemon juice.
	\item Let it aside for 5-10min
	\item Turn on the oven on 450F
	\item Wash chicken on water
	\item Mix the panko breadcrumbs with the italian season.
	\item Brush the chicken on egg whites, then on the panko breadcrumbs.
	\item Put the parchment paper on a tray and then on the oven.
	\item Leave it on the oven for 20-30min or until fully brown.
\end{itemize}

\subsection{Slow Cooked BBQ Ribs}

\paragraph{Ingredients:}

\begin{itemize}
\item Jack Daniels BBQ Sauce 
\item 1 Rack of Ribs 
\item Ribs Seasoning (Safeway would do it)
\end{itemize}

\paragraph{Directions:}
\begin{itemize}
\item Put seasoning in both sides of ribs 
\item Wrap the ribs in aluminum foil 
\item Turn oven on 325F and let it cook for 3-4hrs 
\item Remove aluminum foil and put BBQ sauce 
\item Turn the heat up to 425F and leave for more 10min 
\item Cut and serve
\end{itemize}

\subsection{Ground Beef}

\paragraph{Ingredients:}

\begin{itemize}
	\item 1 onion 
	\item 1 tomato 
	\item 1/4 red pepper 
	\item cilantro 
	\item cumin 
	\item salt 
	\item pepper 
	\item ground beef (Use low fat beef) 
	\item knoor 
	\item tomato paste (1 can)
\end{itemize}

\paragraph{Directions:}
\begin{itemize}
	\item Chop onion, tomatoes, red pepper, cilantro. On a big sauce, put veggies with some olive oil. Stirfor a couple of min until smells. Mix in knoor, beef, cilantro, cummin 
	\item Mix in tomatoe paste 
	\item Season with salt and pepper 
	\item Let it simmer for a good 30min. Be careful for bottom of pan not burn
\end{itemize}

\subsection{Peixe ao coco}

\paragraph{Ingredients:}

\begin{itemize}
	\item Azeite de dende (3 colheres de sopa)
	\item 2 lata de leite de coco (coconut thai)
	\item 1 pimentao vermelho
	\item 1 pimentao laranja
	\item 1 pimentao verde (pequeno)
	\item 1 1/2 cebola
	\item 2.5lb (1.1kg) posta de peixe (tilapia)
	\item 2-3 tomates
	\item coentro
\end{itemize}

\paragraph{Directions:}
\begin{itemize}
	\item Deixe o peixe descansando no limao, sal e pimenta
	\item Refogue o pimentao e a cebola no dende
	\item Quando o pimentao e cebola ficarem macios jogue o tomate e coentro
	\item Preaqueca o forno a 400F
	\item Coloque o leite de coco, deixe aquecer mas n deixe ferver. 
	\item Tempere com sal a gosto
	\item Num refratario coloque o peixe e a mistura anterior
	\item 15'no forno eh o suficiente.
\end{itemize} 

\subsection{Meatballs}

\paragraph{Ingredients (Meatballs):}

\begin{itemize}
	\item 1 pound lean ground beef 
	\item 1/2 cup fresh bread crumbs 
	\item 1 tablespoon grated Parmesan cheese 
	\item 1 teaspoon of onion powder
	\item 1 teaspoon of garlic powder
	\item puree of 4 garlic cloves 
\end{itemize}

\paragraph{Ingredients (Sauce):}
\begin{itemize}
	\item 2 cups of tomato sauce (Cento brand preferably)
	\item some spice: either italian seasoning or chopped parsley
\end{itemize}

\paragraph{Directions:}
\begin{itemize}
	\item Mix all the meatball ingredients and make small meatballs.
	\item Add a bit of olive oil to a pan, sear the meatballs in high heat both sides, then add the tomato sauce and spice, lower the heat down, cover and let it simmer for 10-15min.
\end{itemize}

\section{Sandwich}

\subsection{Mustard Sandwich}

\paragraph{Ingredients:}

\begin{itemize}
	\item 1 tablespoon of mayo 
	\item 1 teaspoon of Dijon mustard 
	\item 2 slices of bread 
	\item 3 slices of ham 
	\item 3 slices of provolone
\end{itemize}

\paragraph{Directions:}
\begin{itemize}
	\item Spread mayo and mustard on the inside part of the bread 
	\item put 3 slices of ham inside the bread 
	\item put 3 slices of provolone in the top of the bread 
	\item put 10min @ 450F on the toaster 
\end{itemize}

\section{Misc}

\subsection{Lemon and Sprite}

\paragraph{Ingredients:}

\begin{itemize}
	\item 2 lemons (the green)
	\item 1 sprite
	\item ice
\end{itemize}

\paragraph{Directions:}
\begin{itemize}
	\item Mix everything
\end{itemize}

\subsection{Caramelized Carrots}

\paragraph{Ingredients:}

\begin{itemize}
	\item 2 chopped carrots
	\item 2 cloves of garlic
	\item 1 tablespoon of honey or agave nectar
	\item 1 orange (Juice the orange)
\end{itemize}

\paragraph{Directions:}
\begin{itemize}
	\item  Heat 2 tablespoons of olive oil on a skillet
	\item Add carrots and garlic cook for 3-5 min until fragrant (you might want to cook carrots on steam before)
	\item Add agave nectar / orange and cook until dries out
	\item Season with salt and pepper to taste
\end{itemize}

\section{Pastas}

\subsection{Spaguetti Squash}

\paragraph{Ingredients:}

\begin{itemize}
	\item Spaguetti Squash
	\item Olive oil
	\item Salt and pepper
	
\end{itemize}

\paragraph{Directions:}
\begin{itemize}
	\item Poke some holes on the squash and put on the microwave for 5'
	\item Remove from the microwave, cut it in half and remove the seeds with a spoon.
	\item Turn oven on 400F and leave it for 35-40'
	\item Using a fork, make the pasta
	\item Add your favourite sauce (marinara, olive oil and garlic, etc.)
\end{itemize}

\subsection{Fettuccine Alfredo (Olive garden)}

\paragraph{Ingredients:}

\begin{itemize}
	\item Parmesan (3 table spoons)
	\item Heavy Cream (1/2 cup)
	\item 1 part of butter
	\item Philadelphia cheese (3 table spoons)
	\item Salt
	\item Pepper
\end{itemize}

\paragraph{Directions:}
\begin{itemize}
	\item Melt butter, then mix all ingredients.
	\item Add salt and pepper to taste
\end{itemize}

\subsection{Carbonara}

\paragraph{Ingredients:}

\begin{itemize}
	\item 2 large eggs
	\item Olive oil
	\item Pancetta or bacon
	\item 4 garlic cloves chopped
	\item 1 cup of parmesan
	\item Ground pepper
	\item Parsley
	\item Spaguetti
\end{itemize}

\paragraph{Directions:}
\begin{itemize}
	\item Mix first the eggs, parsesan and ground pepper (You might add a touch of pecorino cheese as well). Add a pinch of salt.
	\item Heat the skillet and put in either the bacon or pancetta. (if using pancetta you need a bit of olive oil). After the bacon is ready add the garlic for 1min.
	\item When the pasta is ready reserve 1 cup of the water for later
	\item Mix the pasta, and mixture done in step 1 quickly so that the egg cooks (make sure to mix outside the oven so the eggs don’t become scrambled), then mix in the fat, bacon and garlic from step 2. Add the reserved water if necessary.
	\item Season with parmesan, salt and pepper for taste.
	\item Garnish with parsley
\end{itemize}

\section{Healthy}

\subsection{Overnight Oats}

\paragraph{Ingredients:}

\begin{itemize}
	\item 1 cup of old fashioned oats
	\item 1 cup of almond milk
	\item 1 scoop of whey protein
	\item 1/2 tsp of pumping pie spice (1/2 tsp cinnamon, 1/8 tsp nutmet, 1/8 tsp ginger). I actually like more like 8:1:2 the proportions between (cinnamon, nutmeg and ginger).
	\item (Optional) 1 1/2 tsp maple syrup
\end{itemize}

\paragraph{Directions:}
\begin{itemize}
	\item Combine all ingredients, shake well. Leave in the fridge overnight.
\end{itemize}

\subsection{Roasted brussel sprouts}

\paragraph{Ingredients:}

\begin{itemize}
	\item Brussels sprouts
	\item Salt
	\item Pepper
	\item Olive oil
\end{itemize}

\paragraph{Directions:}
\begin{itemize}
	\item Cut brussels sprouts in half, mix with a bit of olive oil, salt and pepper.
	\item Turn oven on 425F and leave for 20'. Flip sides 10'.
\end{itemize}

\subsection{Diced sweet potatoes}

\paragraph{Ingredients:}

\begin{itemize}
	\item 1 lb sweet potatoes
	\item 1 tbl spoon white vinegar
	\item 1 shallot
	\item 1 bunch of chives
	\item Salt
	\item Pepper
	\item Olive oil
\end{itemize}

\paragraph{Directions:}
\begin{itemize}
	\item Thinly slice the chives.
	\item Peel the shallot and mince.
	\item Place chives, shallot and vinegar in a bowl for marinate.
	\item Cook sweet potatoes in boiling water for 10-12min (After peeling and large dicing).
	\item After potatoes are cooked, drain them.
	\item Add olive oil to the same pan, add marinate and sear until onion is translucent.
	\item Add cooked potatoes, mixture and season with salt and pepper.
\end{itemize}

\section{Work in Progress (WIP)}

\subsection{Cauliflower Couscous}

\paragraph{Ingredients:}

\begin{itemize}
	\item 1/2 cauliflower
	\item 3-4 cloves of garlic
	\item Knoor
	\item Seasoning (Curry for example)
\end{itemize}

\paragraph{Directions:}
\begin{itemize}
	\item Grate cauliflower on a food processor for example.
	\item Wrap cauliflower in a towel to remove excess of water
	\item In a sauteed pan, put garlic for 2-3 min (until fragrant)
	\item Add salt, pepper and seasoning (curry for example) and let it cook for 6-7 min (Do not cover with lid otherwise will be watery)
\end{itemize}

\subsection{Creamy Polenta}

\paragraph{Ingredients:}

\begin{itemize}
	\item 1 cup Polenta
	\item 2 tbl spoons butter
	\item 4 tbl spoons powdered milk
	\item 2 tbl spoons of parmesan cheese
	\item 2 cups Whole milk
\end{itemize}

\paragraph{Directions:}
\begin{itemize}
	\item Put milk and polenta into the pan, turn heat to medium-high
	\item Once it starts simmering, turn down to low
	\item Add powdered milk
	\item Season with salt and pepper
	\item When polenta is done (Approximately 10-15') add butter and parmesan cheese.
\end{itemize}

\subsection{Creamy onion Chicken}

\paragraph{Ingredients:}

\begin{itemize}
	\item Garlic 
	\item Green onion 
	\item Chicken 
	\item Powdered Onion Soup (Lipton brand) 
	\item Parmesan cheese 
	\item Mozzarella cheese 
	\item Crème de leite (TODO)
\end{itemize}

\paragraph{Directions:}
\begin{itemize}
	\item Directions Stir fry the chicken with garlic and green onion. 
	\item Than put 1/4 of the package of the onion soup. Stir well.
	\item Then put the chicken with some mozzarella and parmesan cheese on the top. 
	\item Put in the oven on 400F for 10-15min.
\end{itemize}

\subsection{Bread (Rosinha)}

\paragraph{Ingredients:}

\begin{itemize}
	\item 1 cup of sweet potato - cooked and mashed
	\item 1/4 cup of oil
	\item 2 eggs
	\item 1 3/4 cup of gluten free flour (Bob Mills)
	\item 1 cup of brown sugar 
	\item 1/4 tsp of salt
	\item 1 tsp of baking soda (Try baking powder)
	\item 1 tsp of xanthan gum (Bob Mills)
	\item 1/3 cup of water (*try more water)
	\item 1 tsp of cinnamon (*try more)
\end{itemize}

\paragraph{Directions:}
\begin{itemize}
	\item Mix all ingredients, adding the water gently.
	\item Spread a bit of oil on a loaf pan (12 x 4 1/2 inches used in this recipe)
	\item Turn on the oven at 350F for 50'
	\item Check if bread is ready
\end{itemize}

Should yield 20 slices with 100 cal each.

\subsection{Filet Mignon - Luzinete Style}

\paragraph{Ingredients:}

\begin{itemize}
	\item Salt
	\item Pepper
	\item Filet Mignon
	\item Corn starch
	\item Red wine
	\item 2 onions (or 1 large)
\end{itemize}

\paragraph{Directions:}

For leaving overnight:
\begin{itemize}
	\item Wash the meat in water, grate the onion and put 1 cup of red wine. Add salt and pepper to taste.
	\item Put in a container (tupperware) and leave it overnight.
\end{itemize}


For the day the filet is being cooked:
\begin{itemize}
	\item In a large and hot skillet melt a tablespoon of butter
	\item Sear on all sides of the filet
	\item Remove the filet from the skillet, and in the same skillet add the misture (wine and onion) with 2 table spoons of cornstarch (Be careful with the cornstarch so that is fully dissolved)
	\item Let it cook for 5min
	\item Add the filet into the skillet and close it with a lid, put the fire on low and let it cook for 10min.
	\item Turn off the heat, cut the filet in pieces and use the juice from the meat and put back in the skillet. 
	\item Put 1 tablespoon of butter in the skillet and let it cook for a couple more minutes.
\end{itemize}

\subsection{Healthy Chicken Masala}

\paragraph{Ingredients:}

\begin{itemize}
	\item 1 onion chopped
	\item 1 teaspoon ginger
	\item 1 teaspoon coriander
	\item 1 teaspoon garam masala
	\item 1 teaspoon turmeric
	\item 1 teaspoon paprika
	\item 1 teaspoon salt
	\item 1 tablespoon tomatoe paste
	\item 2 chopped tomatoes
	\item 1/2 cup cilantro
	\item 1 tablespoon chopped garlic
	\item 2 chicken breasts
\end{itemize}

\paragraph{Directions:}
\begin{itemize}
	\item Mix all ingredients in the chicken, let it season for a day. 
	\item In a frying pan turn up the heat with oil and sear until ready.
\end{itemize}



\subsection{Chicken Fried Rice}

\paragraph{Ingredients:}

\begin{itemize}
	\item Cooked rice (Leave overnight on the fridge)
	\item 3 tablespoons peanut oil
	\item 2 tablespoons of soy sauce
	\item 1 teaspoon toasted sesame oil
	\item 2 eggs
	\item Chicken
	\item Chickpeas
\end{itemize}

\paragraph{Directions:}
\begin{itemize}
	\item Put peanut oil and crack the eggs
	\item Make an egg scramble then throw in the remaining ingredients
	\item Mix a bit and season with salt if required

\end{itemize}

Note: Experiment with more soy sauce + sesame oil + grated onions

\subsection{Truffle oil fettuccine}

\paragraph{Ingredients:}

\begin{itemize}
	\item Truffle oil 
	\item Parmesan or pecorino cheese 
	\item Butter 
	\item Heavy whipping cream
\end{itemize}

\paragraph{Directions:}
\begin{itemize}
	\item Melt the butter in a pan
	\item Put over heavy whipping cream and parmesan.
	\item Add some salt and pepper and taste accordingly (maybe adding some more parmesan or cream).
	\item Put 1 table spoon of trufle oil in the end. 
\end{itemize}

\section{Deserts}

\newcommand{\mousse}[3]{
\subsection{#1 Mousse}

\paragraph{Ingredients:}

\begin{itemize}
	\item 1 can of condensed milk
	\item 1 can of "Crema de leche"
	\item #2
\end{itemize}

\paragraph{Directions:}
\begin{itemize}
	\item Blend all ingredients on a blender.
\end{itemize}

\paragraph{Note: #3}
}

\mousse{Cheese}{Philadelphia cheese or brazilian requeijao, Guava Jam}{Melt the guava jam and mix after mixing all the ingredients}

\mousse{Passion Fruit}{Passion Fruit Juice}{Melt the guava jam and mix on the mixture after mixing all the ingredients}

\subsection{Creme brulee}

\paragraph{Ingredients:}

\begin{itemize}
\item 1 tsp vanilla bean paste
\item 6 egg yolks
\item 6 tbl spoon of sugar
\item 2 1/2 cup of heavy cream
\end{itemize}

\paragraph{Directions:}
\begin{itemize}
\item Combine heavy cream and sugar in a sauce pan. Place over medium heat, bring just to a simmer and remove from heat.

\item Whisk the egg yolks until they lighten in color in a mixing bowl. Slowly add the cream mixture, mixing continuously.
\item Add back to the sauce pan on medium heat, stir constantly until mixture thickens. About 5'. Don't let it boil.
\item Put mixture in small containers and bake in a preheated oven (300F) for 30'. 
\item Let it refrigerate overnight
\item Using a hand torch add a bit of sugar over the top and caramelize the sugar. Be careful not to burn.
\end{itemize}

Notes: Try with 5 eggs and more vanilla paste (2-3 tsp?)

\subsection{Milk Pudding (Pudim de leite)}

\paragraph{Ingredients:}

\begin{itemize}
	\item 1 can of condensed milk
	\item 1 1/2 cup of sugar
	\item Milk
	\item 1 1/2 spoon of cornstarch
\end{itemize}

\paragraph{Directions:}
\begin{itemize}
	\item Blend the condensed milk, same amount of milk and cornstarch.
	\item On a saucepan, melt sugar. It takes quite a while
	\item In a pudding container, carefully put the melted sugar. It gets solid pretty quickly.
	\item Turn on heat at 350F, setup a bain-marie.
	\item Wrap pudding in aluminium foil, let it cook for 30-40'
	\item Remove aluminium foil and cook until ready (When its not wet inside)
\end{itemize}

\section{Sauces}

\subsection{Tomato Sauce (Sergio)}

\paragraph{Ingredients:}

\begin{itemize}
	\item 1 onion 
	\item 1 tomato 
	\item 3 cloves of garlic 
	\item 1 tomato paste can
\end{itemize}

\paragraph{Directions:}
\begin{itemize}
	\item Chop onion, tomatoes and garlic. On a sauce pan add one table spoon of oil, mix veggies and let it sauté for a while (until smells). 
	\item Remove from sauce pan, put on blender and mix all ingredients. 
	\item Put the blended mix back into the sauce pan. Reduce the fire, mix in tomato paste and let it simmer for 30'.
\end{itemize}

\subsection{Bechamel Sauce (Blue Apron)}

\paragraph{Ingredients:}

\begin{itemize}
\item 1 stick of butter
\item 1 tablespoons all purpose flour
\item 1 cup of whole milk at room temperature
\item 1/2 cup fontina cheese
\item Optional: Nutmeg
\end{itemize}

\paragraph{Directions:}
\begin{itemize}
\item Melt the butter over medium heat
\item Add flour and whisk until smooth (2min)
\item Add the milk gradually
\item Simmer until thick enough (~10min)
\item Optional: Add nutmeg
\item Add the fontina cheese
\item Season with salt and pepper
\end{itemize}

\subsection{Bechamel Sauce (Adapted Luzinete Style)}

\paragraph{Ingredients:}

\begin{itemize}
	\item 2 tablespoons of flour
	\item 1 onion or garlic (2 cloves)
	\item 1 cup of milk
	\item (Optional) pinch of nutmeg
	\item 1 tablespoon of butter
	\item 1 1/2 tablespoon of Parmesan cheese
\end{itemize}

\paragraph{Directions:}
\begin{itemize}
	\item On a blender add the milk and flour
	\item Put the butter on a sautee pan, add grated onion or garlic
	\item Add mixture of milk and flour
	\item Wait until thickens
	\item Add Parmesan cheese and nutmeg
\end{itemize}

\subsection{Bechamel Sauce (Luzinete Style)}

\paragraph{Ingredients:}

\begin{itemize}
	\item 2 tablespoons of flour
	\item 1 onion
	\item 1 cup of milk
	\item 1 tablespoon of butter
	\item 1 1/2 tablespoon of Parmesan cheese
\end{itemize}

\paragraph{Directions:}
\begin{itemize}
	\item On a blender add the onion, milk and flour
	\item Put the butter on a sautee pan, add mixture
	\item Add Parmesan cheese
	\item Wait until it thickens
\end{itemize}

\subsection{Roux}

\paragraph{Ingredients:}

\begin{itemize}
	\item 2 tbl spoon clarified butter (Melted butter where the fat is separated from the milk)
	\item 2 tbl spoon of white flour
\end{itemize}

\paragraph{Directions:}
\begin{itemize}
	\item Melt clarified butter (if not already melted)
	\item Add flour and whisk constantly
	\item For white roux, around 5' will cause flour to lose that raw smell. We are looking for a wet sand consistency. About 20' for blonde roux (smell of toasted bread) and 35' for brown roux (Peanut butter). Dark roux ~ 45'.
\end{itemize}

\subsection{Mother sauces}
\begin{itemize}
	\item Bechamel: White roux mixed with milk
	\item Veloute (From velvet): White roux mixed with clear stock (usually chicken or vegetable).
	\item Espagnole: Dark roux + beef stock.
\end{itemize}

Notes: Mix ingredients (1:8 part ratio). For example: 2 table spoons of roux and 1 cup of liquid (1 cup = 16 tbl spoons), let it simmer for 10' and strain at the end.

\subsection{Pesto Sauce}

\paragraph{Ingredients:}

\begin{itemize}
	\item 2 garlic cloves
	\item 2 cups of basil
	\item 1/4 cup of pine nuts
	\item 1/3 cup of olive oil
	\item 1/2 Parmesan cheese
	\item salt to taste
\end{itemize}

\paragraph{Directions:}
\begin{itemize}
	\item Mix everything in a mixer. Add salt and pepper to taste.
\end{itemize}

Notes: You can make a sauce for pasta by adding a bit of milk and cornstarch (mix both together before putting in pesto mixture).

\subsection{Chipotle Mayo}

\paragraph{Ingredients:}

\begin{itemize}
	\item 2 eggs raw
	\item oil (Canola or Peanut oil)
	\item Chipotle peppers in adobo sauce (in the can)
	\item salt
	\item lime
\end{itemize}

\paragraph{Directions:}
\begin{itemize}
	\item Using a mixer, add the two eggs and oil until it emulsifies. Don't be shy on the oil.
	\item Use a spoon to see if the consistency reminds you of mayonnaise.
	\item Add chipotle peppers (You might want to half first otherwise it might be too spicy)
	\item Add the lime of 2 lemon wedges
	\item Add salt to taste
\end{itemize}

\section{Salad Related}

\subsection{Miso Dressing}

\paragraph{Ingredients:}

\begin{itemize}
	\item 1/3 cup of lemon juice
	\item 1/2 cup of olive oil
	\item 1 tbl spoon of miso paste
	\item pinch of salt
	\item 2 garlic cloves finely minced
\end{itemize}

\paragraph{Directions:}
\begin{itemize}
	\item Mix all ingredients
\end{itemize}

\subsection{Cesar Dressing (Blue Apron)}

\paragraph{Ingredients:}

\begin{itemize}
	\item 1 teaspoon of lemon zest
	\item 2 lemon wedges (the juice)
	\item 2 garlic cloves
	\item 1 tbl spoon white vinegar
	\item 1/4 cup mayo
\end{itemize}

\paragraph{Directions:}
\begin{itemize}
	\item Smash garlic until it resembles a paste, add vinegar and place it in a bowl. Let it marinate for 5-10'. 
	\item Mix all ingredients.
	\item Season with salt and pepper.
\end{itemize}

\subsection{Potato Salad}

\paragraph{Ingredients:}

\begin{itemize}
	\item 5 large potatoes 
	\item 2 ribs of celery, finely chopped 
	\item 1/2 onion, finely chopped 
	\item 3 hard boiled eggs. 2 chopped, 1 sliced 
	\item 1 cup miracle whip 
	\item 3 tablespoons Dijon mustard 
	\item salt 
	\item pepper
\end{itemize}

\paragraph{Ingredients (Miracle Whip):}

\begin{itemize}
	\item 6 teaspoons of white vinegar 
	\item 2 teaspoons of cornstarch 
	\item 3 teaspoon of sugar 
	\item 1 teaspoon paprika 
	\item 1 teaspoon of garlic salt 
	\item 1/4 teaspoon of mustard powder 
	\item 1 1/2 cup of mayo
\end{itemize}

\paragraph{Directions:}
\begin{itemize}
	\item Mix everything and season with salt and pepper to taste.
\end{itemize}

\section{Dump}

\subsection{Rosbife Tia Tina (Tia Ana)}

\paragraph{Ingredients:}

\begin{itemize}
	\item manteiga
	\item limao
	\item pimenta branca
	\item sal
	\item alho
	\item molho ingles
\end{itemize}

\paragraph{Directions:}
\begin{itemize}
	\item Faz uns furinhos na carne pra deixar o tempero entrar. depois de algumas horas temperando eh so por na panela manteiga deixar bem quente e depois jogar o file.
	\item Sela o file numa panela aderente. Fica esfregando o file na panela. So vira o file quando tiver queimado de um lado. O segredo eh ficar esfregando o file na frigideira.
	\item Pra fazer o molho joga um pouco de molho ingles e agua e mistura as raspas que ficam na panela. prova e ve se precisa de sal. Fica provando pra ver se precisa de mais molho ingles.
\end{itemize}

\subsection{Stir Fried Chayote Squash (Sergio)}

\paragraph{Ingredients:}

\begin{itemize}
	\item 1 onion
	\item 3 garlic cloves
	\item 1 tomato
	\item Chayote Squash
	\item knoor 
\end{itemize}

\paragraph{Directions:}
\begin{itemize}
	\item Stir fry the onion and garlic with some oil until garlic is fragrant
	\item Add tomato and knor
	\item Add Chayote Squash (after being steamed)
\end{itemize}

\end{document}