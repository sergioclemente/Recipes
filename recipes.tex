\documentclass{article}

\usepackage{listings}

\usepackage[a4paper, margin=0.8in]{geometry}

\title{Recipes}
\author{Sergio Clemente Filho}
\date{\today}

\begin{document}

\maketitle

\newpage

\tableofcontents{}

\newpage

\section{Flavor Profiles}
\begin{itemize}
	\item Sweet
	\item Sour
	\item Salty
	\item Spicy
	\item Bitter
\end{itemize}

	\section{Ratios}
\begin{itemize}
	\item Ratio of oil to vinegar: 4/1 or 2/1 to more sour resistants.
\end{itemize}

\section{General Tips}
\begin{itemize}
	\item Cut perpendicular to the fiber to maximize tenderness (Meats)
\end{itemize}

\newcommand{\chick}[3]{
\subsection{[TODO: Add Title]}

\paragraph{Ingredients:}

\begin{itemize}
\item #2
\item 1 Chicken Breast 
\item oil
\end{itemize}

\paragraph{Directions:}
\begin{itemize}
\item Cut the chicken, wash and add some lime. Alternatively: Just washing the chicken and patting with paper towels for removing excess moisture (After last step you can optionally coat chicken with flour for making it crisp).
\item If you let it \textbf{lime}, let it sit for 10min, than wash the chicken with running water
\item (Optional) If you can put all ingredients in a ziplock bag and let it rest for couple of hours or overnight.
\item On a sautee pan, add some oil and sear one side of the chicken in high heat.
\item Throw in the remaining ingredients. Lower the heat, flip the side of the chicken, add a lid and let it simmer for 10-15min.
\item If you have a temperature probe, make sure its 165F
\end{itemize}

\paragraph{Notes:} #3
}

\newcommand{\placeholder}[2]{
	\subsection{[TODO: Add Title]}
	
	Fill it
	
		#2 
}

\section{Fast everyday recipes}

\chick{Curry Chicken}{1 tsp knoor, 2 tbl spoon curry powder, 1 tbl spoon Coconut oil}{}

\chick{Galinha Baiana}{Tempero baiano, 3-4 colheres de molho ingles, 6-7 dentes de alho amassado, 2 colheres de molho shoyu, 3 colheres de molho de tomate, 1 colher de cha de knoor}{}

\chick{Galinha italiana}{Cuminho, Oregano, Alho e sal}

\chick{Cilantro chicken}{Cilantro, Chopped onion and Salt}

\chick{Chicken on Tomato Sauce}{1 can of diced tomato (fire roasted), cummin, cilantro, 1 knoor, chopped garlid}

\chick{Onion chicken}{2 onions, 2 tsp of chopped garlic, 2 tablespoons of olive oil, 1 knoor}{User a blender to blend onion and garlic. Use just enough water so you can get a pasty texture}

\chick{Curried Tomato}{2 onions, 5 chopped tomatoes, 1 tablespoon of garlic, 1 tsp cumin, 2/3 tsp paprika, 2/3 tsp curry, 1 knoor}{User a blender to blend onion and garlic.}

\subsection{Soy-Glazed Beef}

\paragraph{Ingredients:}

\begin{itemize}
	\item beef for beef stew (1lb)
	\item 1 onion
	\item 2-3 garlic cloves (1 table spoon of chopped garlic)
	\item 1/4 red pepper (Optional)
	\item 1/4 tea spoon ginger powder (Optional)
	\item 1 cup soy sauce (low sodium)
\end{itemize}

\paragraph{Directions:}
\begin{itemize}
	\item Add a bit of oil to a saute pan, add onion and red pepper for 2-3min, then add garlic and pepper. Then add the garlic until its fragrant.
	\item Reserve the onion and pepper.
	\item On the same pan add the beef, sear the beef all sides
	\item When beef is seared (4-5min) add the soy sauce, optionally the ginger and let it cook for 10-15min.
\end{itemize}

\paragraph{Notes:}You can add some sesame oil on the pan, just when it gets done.

\subsection{Leozin’s Meat Marinade}

\paragraph{Ingredients:}

\begin{itemize}
   \item 2 lb of meat
	\item 8 table spoon soy sauce
	\item 6 table spoon of molho ingles
	\item 4 minced garlic
	\item pepper to taste
	\item salt 
\end{itemize}

\paragraph{Directions:}
\begin{itemize}
	\item Mix all ingredients and let if marinate for a few hours
	\item Stir fry on a pan with some oil
\end{itemize}

\subsection{Grandma’s Chicken}

\paragraph{Ingredients:}

\begin{itemize}
	\item 1 cebola 
	\item 1 tomate 
	\item 1/2 pimentao 
	\item cebolinha (o verde e branco tudo junto) 
	\item 2-3 dentes de alho 
	\item Coentro 
	\item 1:1 caldo knoor pra peito de frango. E.g. 2 caldos knoors para 2 peitos de frango
	\item Molho de tomate a gosto (1/2 xicara)
	\item Vinagre branco a gosto (1-2 colheres)
\end{itemize}

\paragraph{Directions:}
\begin{itemize}
	\item Refoga com azeite os ingredients.
	\item Corta a galinha em pedacos e coloca na agua e limao. Depois de lavar a galinha,
	\item Joga a galinha no refogado
	\item Joga o cuminho com sal.
	\item Bota molho de tomate e vinagre (1 colher)
	\item Deixa por uns 30-40min em fogo baixo tampado.
\end{itemize}

\paragraph{Note}
\begin{itemize}
	\item Pra um sabor diferente use 2 colheres de molho ingles e 2 colheres de molho soyu.
\end{itemize}

\subsection{Oven Roasted Chicken}

\paragraph{Ingredients (Chicken):}
\begin{itemize}
	\item Half chicken
\end{itemize}

\paragraph{Directions (Chicken):}
\begin{itemize}
	\item Preheat the oven to 400°F. 
	\item Line a sheet pan with aluminum foil. 
	\item Pat the chicken dry with paper towels; season with salt and pepper on
	both sides. Place on the prepared sheet pan. Drizzle with olive oil; turn
	to coat. Arrange the seasoned chicken skin side up. 
	\item Roast 36 to 38 minutes, or until the chicken is browned and cooked
	through. (An instant-read thermometer inserted into the thickest part
	of the thigh should register 165ºF.) Remove from the oven. 
	\item Reserving the juices on the sheet pan, transfer the roasted chicken to a
	cutting board. Let rest for at least 5 minutes.
\end{itemize}

\paragraph{Ingredients (Gravy):}
\begin{itemize}
	\item 2 tbl spoon flour
	\item 2 tbl spoon butter
\end{itemize}

\paragraph{Directions (Gravy):}
\begin{itemize}
	\item Heat 2 tablespoons of butter and add 2 tablespoons of flour
	\item Add chicken juices from cooked chicken (Separate butter)
\end{itemize}

\subsection{Pork in Mustard and Vinegar Sauce}

\paragraph{Ingredients:}

\begin{itemize}
	\item 1/2 onion
	\item 1 clove of garlic
	\item 1 table spoon mustard
	\item 3 table spoons of apple cider vinegar
	\item 3 tablespoons oil or olive oil
	\item 1 tablespoon margarine
	\item salt
	\item pepper
\end{itemize}

\paragraph{Directions:}
\begin{itemize}
	\item Chop onion
	\item Mince garlic
	\item Mix onion, garlic, mustard, apple cider vinegar and put into a bowl
	\item Put pork into bowl and let it sit for 30'
	\item On a skillet put margarine and a bit of oil
	\item Add pork
	\item Season both sides, then pour the onion mixture
	\item Let it cook a bit and done
\end{itemize}

\subsection{Mexican Carne Assada}

\paragraph{Ingredients:}

\begin{itemize}
	\item 1 cut of flank steak
	\item 1 lemon
	\item 4 garlic cloves
	\item 1/2 cup of soy sauce
	\item 1 teaspoon chilli powder
	\item 1 teaspoon paprika
	\item 1 teaspoon ground cumin
	\item 1 teaspoon oregano
	\item black pepper
	\item olive oil
\end{itemize}

\paragraph{Directions:}
\begin{itemize}
	\item Mix all ingredients
	\item Let it sit in the fridge overnight
	\item On a high heat, put meat first
	\item Then add the juices and sauce
	\item Simmer for 10-15'
\end{itemize}

\section{Jane Loures}

\subsection{Homemade Cumin Seasoning Mix}

\paragraph{Ingredients:}
\begin{itemize}
	\item 6 dentes de alho
	\item 100ml de oleo
	\item 1/2 cebola
	\item 1/4 colher de cha de cominho
	\item 2 colheres de molho ingles
	\item 2 colheres de azeite
	\item 2 colheres de sal
\end{itemize}

\paragraph{Directions:}
\begin{itemize}
	\item Bate tudo no liquidificador. Vai virar uma manteiga.
\end{itemize}

\subsection{Bife acebolado}

\paragraph{Ingredients:}
\begin{itemize}
	\item 400-600g de carne
	\item Tempero caseiro a vontade
	\item Molho ingles  avontade
	\item 1/2 colher de cha de pimenta do reino em po
	\item 1 cebola grande cortada em rodelas.
\end{itemize}

\paragraph{Directions:}
\begin{itemize}
	\item Coloca o tempero caseiro, molho ingles e pimenta na carne. 
	\item Massageia a carne pra o tempero entrar.
	\item Coloca o oleo na figideira para esquentar.
	\item Sela a carne dos dois lado, depois que a carne ta boa reserva.
	\item Frita a cebola.
	\item Adiciona a carne de volta e adiciona um copo de agua (200ml) com uma colher de cha de tempero caseiro.
\end{itemize}

\section{Starches / Beans}

\subsection{Simple Rice}

\paragraph{Ingredients:}

\begin{itemize}
	\item 1 cup of rice
	\item 2 cups of water
	\item 1 table spoon of knoor
\end{itemize}

\paragraph{Directions:}
\begin{itemize}
	\item Boil water, then bring it to a simmer
	\item Add rice and knoor
	\item Cook covered for 20-30'
	\item Turn heat off (don't uncover) and let it sit for 20'
\end{itemize} 

PS: Couple of variants of this rice: - Chop 1/2 onion and 2 garlic cloves. Add 1 tbsp of oil and stir fry the onion and garlic until garlic gets fragrant and onion translucent. Add rice and stir for 1-2’, then do the same
steps in the Directions.

\subsection{Potatoes Gratin (Sergio)}

\paragraph{Ingredients:}

\begin{itemize}
	\item Parmesan
	\item Whipping Cream
	\item Cream Cheese Whipped
	\item Mayo
	\item Oregano
	\item Musarella Cheese
	\item Potatoes
\end{itemize}

\paragraph{Directions:}
\begin{itemize}
	\item Cut the potatoes in small cubes and boil until is al dente.
	\item Mix all the other ingredients, mix with the potatoes. Top with musarrela cheese and put in the oven on 450F for 30min.
\end{itemize}

\subsection{Refried beans}

\paragraph{Ingredients:}

\begin{itemize}
	\item Already made beans (2 cups)
	\item Mexican Spice Blend (2 tsp paprika, 2 tsp chili powder, 1/2 tsp garlic salt, 1/2 tsp onion salt, 1 tsp cumin, pepper to taste)
\end{itemize}

\paragraph{Directions:}
\begin{itemize}
	\item Put beans on a blender and blend until forms a puree
	\item On a sauce pan add mexican spice blend.
	\item Reduce heat, let it reduce. 5'.
\end{itemize} 

\subsection{Brazilian "Risotto"}

\paragraph{Ingredients:}

\begin{itemize}
	\item Arroz
	\item 1 colher de trigo
	\item 1 xicara de leite
	\item 1 colher de Queijo parmesao
	\item Pimenta branca
	\item 1 colher de sobremesa de manteiga
\end{itemize}

\paragraph{Directions:}
\begin{itemize}
	\item Cozinhe o arroz normal
	\item Faca um white Roux
	\item Adicione o leite
	\item Coloque o parmesao e a pimenta branca
	\item Pode colocar opcionalmente um queijo musarela
\end{itemize} 

\subsection{Brazilian Feijoada}

\paragraph{Ingredients:}

\begin{itemize}
	\item 1lb of Black beans 
	\item 12oz Salted Pork(Hormel Salt Pork, Cured, Sliced)
	\item 1 Calabresa Sausage (Hillshire Farm Smoked Sausage)
	\item 1 piece of chopped pork loin (or another pork cut)
	\item 1 tsp sweet paprika
	\item 1 tsp cumin
	\item 1-2 Bay leaves 
	\item 3oz Tomato paste (Hunts Tomato Paste 6oz). UPDATE: I've recently updated to use a table spoon only.  I think it tastes much better.
	\item 1/3 of bunch of cilantro (without the stems)
	\item 1 onion 
	\item 1 tomato 
	\item 1/4 red pepper 
	\item 1 table spoon white vinegar
\end{itemize}

\paragraph{Directions:}
\begin{itemize}
	\item Sautee the meat with Cumin, Paprika, garlic and vinegar (a bit) and salt
	\item Add tomato, onion, red pepper, cilantro, tomato paste and water to a blender. Add salt.
	\item Put beans in water for 3 hrs.
	\item Mix beans, mixture and bay leaves in the pressure cooker.
	\item Cook under pressure for 10', then add the meat and cook for another 5-10' until beans are tender.
	\item Season with salt and pepper if necessary.
	\item If beans are too watery you can always add some beans to the blender to make a puree and add it back.
\end{itemize}

\paragraph{Notes:}
Easy mode of this dish is put everything in the slow cooker (high mode) for 5hrs.

\subsection{Creamy Polenta}

\paragraph{Ingredients:}

\begin{itemize}
	\item 1 cup Polenta. The finer grain the better. Been relatively successful with Delallo Instant Polenta.
	\item 1 tbl spoons butter
	\item 2 tbl spoons powdered milk
	\item 2 tbl spoons of parmesan cheese
	\item 2 cups Whole milk
\end{itemize}

\paragraph{Directions:}
\begin{itemize}
	\item Add milk and polenta into the pan, turn heat to medium-high
	\item Once it starts simmering, turn down to low
	\item Add powdered milk
	\item Wish constantly
	\item Season with salt and pepper
	\item When polenta is done (Approximately 10-15’) add butter and parmesan cheese.
\end{itemize}

\subsection{Loaded Scalloped Potatoes}

\paragraph{Ingredients:}
\begin{itemize}
\item 4 tbsp butter unsalted
\item 3 cloves garlic minced
\item 4 tbsp all-purpose flour
\item 2 cups chicken broth low sodium
\item 1 cup heavy cream
\item 1/2 tsp salt or to taste
\item 1/2 tsp pepper or to taste
\item pinch tsp nutmeg ground
\item 3 lbs potatoes sliced 1/8 inch thin, I used Yukon gold
\item 2 cups cheddar cheese shredded (you can complement with mozzarella too)
\item 8-12 slices bacon fried and crumbled
\item 2 tbsp chives chopped
\end{itemize}

\paragraph{Directions:}
\begin{itemize}
\item Preheat your oven to \textbf{400 F} degrees.
\item Melt the \textbf{butter} in a saucepan or skillet over medium heat. Add the \textbf{garlic} and cook for 30 seconds until
the garlic is aromatic.
\item Sprinkle the \textbf{flour} over the garlic/butter then whisk well until combined with the butter. Add the \textbf{chicken broth} and whisk until combined. It should look like a thick paste. Cook for another minute while stirring, then add the \textbf{heavy cream} and whisk until smooth.
\item Season with \textbf{salt, pepper and nutmeg}. Continue cooking for 2 more minutes until the sauce comes to a simmer and thickens. Remove the saucepan from the heat and set aside.
\item Start layering. Start with a spreading a couple ladlefuls of the sauce over the bottom of a 9x13 inch baking dish. Spread a third of the sliced potatoes over the sauce in an even layer (see video). Top with
more of the sauce to cover the potatoes. Top with a third of the cheddar cheese and then a third of the bacon. You should have enough ingredients for 3 layers of potatoes. Continue layering with potatoes, sauce, cheese, bacon and repeat. Finally sprinkle with \textbf{half the chives}.
\item Cover with foil and bake for \textbf{45 minutes}. Uncover and finish baking for for \textbf{another 45 minutes} or until the potatoes are fork tender.
\item Garnish with remaining \textbf{chives} and serve warm.
\end{itemize}

\subsection{Beans in Tomato Sauce (Luzinete Style)}

\paragraph{Ingredients:}

\begin{itemize}
	\item 2 cups of soybeans/white beans
	\item olive oil
	\item 1 onion
	\item 1 tomato
	\item 1 cup of Tomato sauce (The brazilian one Tarantella is the best)
\end{itemize}

\paragraph{Directions:}
\begin{itemize}
	\item On a sautee pan cook chopped onions and tomato
	\item Cook the soybeans on the pressure cooker (w/ onion an tomato) with just water and salt for 20/40' (Depending whether is white beans/soy).
\end{itemize}

\subsection{Truffle Sauce for Pasta}

\paragraph{Truffle butter}
\begin{itemize}
	\item Put two blocks of butter with a truffle in a zip-lock bag. Let it rest for 2 days in the fridge.
\end{itemize}

\paragraph{Ingredients:}

\begin{itemize}
	\item 4 tbl spoons of truffle butter
	\item 1 truffle shredded (size of a brussel sprouts)
	\item 1/4 cup of milk
	\item 1/3 cup of shredded cheese (used Beecher's. you can use parmesan too)
\end{itemize}

\paragraph{Directions:}
\begin{itemize}
	\item Heat butter in as saucepan
	\item Add shredded truffle
	\item Stir until butter is fully melt
	\item Add cheese
	\item Add milk slowly, taste, you might need more or less milk if truffle is too strong.
\end{itemize}

\subsection{Sushi Rice}

TODO

\paragraph{Ingredients:}

\begin{itemize}
	\item
\end{itemize}

\paragraph{Directions:}
\begin{itemize}
	\item
\end{itemize}


\section{Bacalhau}

\subsection{Bacalhau à Gomes de Sá (Tia Ana)}

\paragraph{Ingredients:}

\begin{itemize}
	\item Bacalhau (1.5kg)
	\item Batata (6 batatas)
	\item Cebola (6 cebolas)
	\item Alho (1 cabeca de alho)
	\item Azeite (2 latas de azeite galo)
	\item Tomate (4 tomates)
	\item Ovo (8 ovos)
	\item Salsa (1 bocadinho)
\end{itemize}

\paragraph{Directions:}
\begin{itemize}
	\item Doura cebola e alho no azeite (Bastante azeite). Nao colocar no oleo quente. Deixar a cebola e o alho cozinhar no azeite.
	\item Cozinha o bacalhau na agua e sal por pouco tempo, separa e desfia (Se o bacalhau ja for salgado não precisa colocar mais sal
	\item Cozinha batata cortada em cubinhos. Nao deixar cozinhar muito pra não virar um "pure" quando misturar no bacalhau depois
	\item Quando a cebola comecar a dourar, joga o peixe e deixa uns 10min
	\item Coloca salsa picada
	\item Coloca a batata e espera ver se vai precisar botar mais azeite
	\item Coloca ovo cozido amassado no bacalhau (so coloca 4) os outros voce coloca fatiado por cima
	\item Deixa descansar durante a noite (pode botar em tupperware desde que esteja frio).
	\item No dia seguinte corta umas rodelas de ovo cozido e poe por cima e umas fatias de tomate tambem. Bota no forno pra deixar  queimado em cima.
\end{itemize}

\subsection{Bacalhau Espiritual - Modificado}

\paragraph{Ingredients:}

\begin{itemize}
	\item 1 lb bacalhau
	\item Azeite de oliva
	\item 1 onion
	\item 2 garlic cloves crushed
	\item 10 yellow potatoes, chopped
	\item Bechamel Sauce (Blue Apron)
\end{itemize}

\paragraph{Directions:}
\begin{itemize}
	\item (T-2) Deixar o bacalhau de molho, trocar a agua pelo menos 3 vezes.
	\item (T-1) Cozer o bacalhau, depois que estiver fervendo contar 5 minutos. Retire do fogo. Descarte a agua e desfie.
	\item (T-1) Fritar a cebola no azeite, adicionar o alho e logo em seguida o bacalhau desfiado.
	\item (T-1) Deixar o bacalhau fritando um pouco (uns 5minutos)
	\item (T) Corte a batata, leve para ferver. Quando estiver ao dente retire a batata do forno.
	\item (T) Fazer o molho branco seguindo a receita to Molho Bechamel (Blue Apron) ao pe da letra.
	\item (T) Misture tudo (Molho Branco, Batata e base do bacalhau)
	\item (T) Adicione mais queijo fontina em cima a gosto e leve ao forno para gratinar (450F)
\end{itemize}

\paragraph{Notes:}
A base desse bacalhau eh o gomes sa + o molho bechamel. A grande diferenca do gomes sa eh que essa receita leva bem menos azeite ja que vai levar o molho branco. A receita original [1] leva pao, mas eu acho melhor com batata.

[1] https://www.196flavors.com/portugal-bacalhau-espiritual/

Essa receita demora 3 dias para ficar pronta:
\begin{itemize}
	\item T-2: Bacalhau fica de molho pra desalgar
	\item T-1: Faz a base do gomes sa
	\item T: Faz o molho branco, batata e mistura com o bacalhau
\end{itemize}

\subsection{Chef Zeca – Creme de Bacalhau}

\paragraph{Ingredients:}

\begin{itemize}
	\item 2lb of bacalhau
	\item 1 onion
	\item 1 red bell pepper cut in thin strips
	\item 1 yellow bell pepper cut in thin strips
	\item 5 garlic cloves minced
	\item 2 leeks cut in small pieces (sliced then in 4 pieces each cilinder)
	\item cream cheese (catupiry works better)
	\item 1 can of creme de leche
	\item 1/2lb of parmesan cheese
	\item olive oil
	\item salt
	\item ground pepper to taste
\end{itemize}

\paragraph{Directions:}
\begin{itemize}
	\item Prepare the bacalao like the other recipes until the boiling phase.
	\item Add olive oil wait it to get warm ~275F-300F
	\item Add onions for 3min or so
	\item Season with ground pepper
	\item Add bell pepper
	\item Add garlic and leeks
	\item sear for a few minutes then add the bacalhao (should be boiled and in small bits)
	\item Mix it thoroughly
	\item Add more olive oil, then wait for a few 2-3 minutes
	\item Add cheese and creme de leche
	\item Taste for salt
	\item Plate the bacalao and add parmesan cheese
	\item Put in the oven @ 400F until parmesan crust shows up
\end{itemize}

\paragraph{Notes:}

Fazer esse bacalhau inicialmente como os anteriores, deixar de molho na agua e dar uma leve fervura no dia seguinte.

https://www.youtube.com/watch?v=sf8dNmpWWVA

\subsection{Chef Zeca - Bacalhau ao Ze do pipo}

\paragraph{Ingredients:}

\begin{itemize}
	\item 2lb of bacalao
	\item 1 Red onion sliced
	\item 5 garlic cloves sliced
	\item Green bell pepper sliced in thin strips
	\item Red bell pepper sliced in thin strips
	\item 1 bay leaf
	\item Mayonnaise
	\item black olives sliced in half without seeds
	\item Mashed potatoes Recipe
\end{itemize}

\paragraph{Directions:}
\begin{itemize}
	\item Prepare the bacalao like the other recipes until the boiling phase.
	\item Add 3 table spoons olive oil wait it to get warm ~275F-300F
	\item Add onions for 3min or so
	\item Season with ground pepper
	\item Add the garlic
	\item Add bacalao
	\item Taste for salt
	\item add bay leaf
	\item Add bacalao mixture at the bottom
	\item Add mashed potatoes on top
	\item Add Mayonnaise on top
	\item Decorate with the bell peppers and olives
	\item 400F in the oven until it gets lightly brown
\end{itemize}

\paragraph{Notes:}

Fazer esse bacalhau inicialmente como os anteriores, deixar de molho na agua e dar uma leve fervura no dia seguinte.

https://www.youtube.com/watch?v=vKF096k3oxk

\subsection{Chef Zeca – Bacalhau à Gomes de Sá}

\paragraph{Ingredients:}

\begin{itemize}
	\item 2lb of potatoes in small cubes (ratio between bacalao and potatoes we are looking for is 1:1)
	\item 3 red onions
	\item 1/2 red pepper
	\item 1/2 yellow pepper
	\item 1/2 lb of black olives in slices
	\item 8 garlic cloves
	\item 2 lb of bacalao
	\item 3 eggs bolied in slices
	\item bay leaf
	\item olive oil
	\item salt
	\item pepper
\end{itemize}

\paragraph{Directions:}
\begin{itemize}
	\item Boil water then add potatoes
	\item Turn off the heat and drain water when potatoes are al dente
	\item Prepare the bacalao like the other recipes until the boiling phase.
	\item Add 3 table spoons olive oil wait it to get warm ~275F-300F
	\item Add onions for 3min or so
	\item Add peppers
	\item Season with ground pepper
	\item Add garlic and bay leaves
	\item Add bacalao (Should be in small pieces)
	\item Add more olive oil
	\item Add black olives and more olive oil
	\item Add potatoes in the bottom
	\item Add bacalao mixture
	\item Add more bacalao mixture
	\item Add eggs
	\item 400F in the oven until it gets lightly brown
\end{itemize}

\paragraph{Notes:}

Fazer esse bacalhau inicialmente como os anteriores, deixar de molho na agua e dar uma leve fervura no dia seguinte.

https://www.youtube.com/watch?v=biRLy3Uuj3Q

\subsection{Chef Zeca - Bolinho de bacalhau}

\paragraph{Ingredients:}

\begin{itemize}
	\item 1lb of bacalao
	\item 1lb of asterix potatoes (1:1 ratio), boil them and mash like a puree
	\item 1 red onion in small pieces
	\item green onion in small slices
	\item parsley in small pieces
	\item 3 garlic cloves in small pieces
	\item 2 eggs
	\item lemon
	\item panko breakcrumb (3/4 of a bowl)
	\item olive oil
	\item black pepper
\end{itemize}

\paragraph{Directions:}
\begin{itemize}
	\item Prepare the bacalao like the other recipes until the boiling phase.
	\item Add 3 table spoons olive oil wait it to get warm ~275F-300F
	\item Add onions for 3min or so
	\item Season with ground pepper
	\item Add garlic
	\item Add bacalao (Should be in small pieces)
	\item Taste for salt
\end{itemize}

\paragraph{Directions (For balls)}
\begin{itemize}
	\item In a bowl, mix puree, bacalao mixture and green onion and parsley
	\item Add lemon
	\item Taste for salt
	\item add  eggs
	\item Add half the panko 
	\item Make small ball shaped (~1oz per ball)
	\item Coat each ball with panko
\end{itemize}



\paragraph{Directions (For frying)}
\begin{itemize}
	\item Add oil to a pan until its hot ~350F
	\item Add the balls in a controlled manner to the frying pan so that the temperature doesn't drop too much
	\item Fry the balls like you would deep fry other stuff like french fries
\end{itemize}

\paragraph{Notes:}
\begin{itemize}
	\item If you want to freeze you can stop here. If you froze you can continue from here. If it was frozen, wait 30min before throwing them in the frying pan
	\item Fazer esse bacalhau inicialmente como os anteriores, deixar de molho na agua e dar uma leve fervura no dia seguinte.
	\item https://www.youtube.com/watch?v=tA48tXulepQ
\end{itemize}

\section{Bread \&t Pizza}

\subsection{Bread}{}

\paragraph{Ingredients:}

\begin{itemize}
	\item 800g flour
	\item 540g water
	\item 25g salt
	\item 7g dry yeast
\end{itemize}

\paragraph{Directions:}
\begin{itemize}
	\item Add water and mix ingredients
	\item Fold dough 3-5 times every 30min.
	\item Let it rise for 2-3 hours
	\item On dutch oven (using the water inside the dutch oven) cook for 30min @ 480F with the lid closed
	\item Open the lid to allow bread to brown
\end{itemize}

https://www.youtube.com/watch?v=3Uc8h4T7GDo

\subsection{Pizza Dough}


\paragraph{Ingredients:}

\begin{itemize}
	\item 250g of 00 flour
	\item 5g of salt
	\item 1.5 active dry yeast
	\item 162.5g of warm watter
\end{itemize}


\paragraph{Directions:}
\begin{itemize}
	\item mix all ingredients (be-careful with mixing salt and yeast together as the salt kills the yeast)
	\item turn the light of the oven on, put a wet cloth
	\item leave rising for a few hours
\end{itemize}

\section{Smoker}

\subsection{Smoked Salmon}{}

\paragraph{Ingredients:}

\begin{itemize}
	\item 1 : 2 mixture between salt to sugar (e.g. 2 cups of salt to 4 cups of sugar typically handles 2-3 trays for fish)
	\item Fresh ground pepper
\end{itemize}

\paragraph{Directions:}
\begin{itemize}
	\item Spread cure on both sides and inside of the fish and use a plastic wrap around it.
	\item Transfer to refrigerator and cure for 4-6 hours
	\item Wash the fish
	\item Set smoker temperature to 180F
	\item Use temperature probe to make sure fish is ~140F
\end{itemize}


\subsection{Pulled Pork}{}

\paragraph{Ingredients:}

\begin{itemize}
	\item 1 bone in pork shoulder
	\item Pork rub
	\item BBQ Sauce
	\item 2 cup apple cider
\end{itemize}

\paragraph{Directions:}
\begin{itemize}
	\item Rub season on pork and let it sit for 20min
	\item Turn on Smoker at 250F
	\item Put pork on grill with butt side up until temp reaches 160F
	\item Pour apple cider into pork
	\item Wrap pork on aluminum foil
	\item Add back to the grill until internal temp reaches 204F
\end{itemize}

\section{Seafood}

\subsection{Crab Cakes}

TODO

\paragraph{Ingredients:}

\begin{itemize}
	\item
\end{itemize}

\paragraph{Directions:}
\begin{itemize}
	\item
\end{itemize}


\subsection{Salmon Burger}

TODO

\paragraph{Ingredients:}

\begin{itemize}
	\item
\end{itemize}

\paragraph{Directions:}
\begin{itemize}
	\item
\end{itemize}


\subsection{Fish Tacos}{}

\paragraph{Ingredients (Sour cream):}
\begin{itemize}
	\item 1:2 ratio between sour cream and mayo (e.g. 1/2 cup of sour cream and 1/4 cup of mayo)
	\item juice from 1 lime
	\item salt and pepper to taste
\end{itemize}

\paragraph{Ingredients (Guacamole):}
\begin{itemize}
	\item 2 avocados smashed
	\item juice from 1 lime
	\item salt and pepper to taste
\end{itemize}

\paragraph{Ingredients (Pico de gallo):}
\begin{itemize}
	\item 2 tomatoes chopped
	\item 1 red onion
	\item 1/2 cup of chopped cilantro
	\item salt and pepper to taste
\end{itemize}

\paragraph{Ingredients (Main):}
\begin{itemize}
	\item 1-2lb Fish
	\item Tortilhas
\end{itemize}

\paragraph{Directions:}
\begin{itemize}
	\item
\end{itemize}

\subsection{Moqueca}

\paragraph{Ingredients:}

\begin{itemize}
	\item 1 onion	
	\item 1/2 red bell pepper 
	\item 1/2 yellow bell pepper 
	\item 2 tomatoes
	\item 4 garlic cloves
	\item 150ml of coconut milk (CHAOJKOH)
	\item 2 table spoon of chopped cilantro
	\item tomato sauce
	\item 2-3 tbl spoon red palm oil (NUTIVA)
	\item 2-3 tbl spoon olive oil
	\item salt
	\item black pepper
\end{itemize}

\paragraph{Directions:}
\begin{itemize}
	\item Add olive oil + red palm oil in a pan, wait til is warm
	\item Add onions for 2-3min, then bell peppers for 2-3min, then garlic, add tomatoes and cilantro
	\item Add Coconut milk (Maybe less than 150ml)
	\item Season with salt, then add tomato sauce til color is somewhat light yellow/orange.
	\item Let is cook in low heat for a few minutes
\end{itemize}

\paragraph{Notes:} You can use crab or fish here, insert the crab or fish just before you add the coconut milk. For the crab you want to take off the shell and for the fish you want to a light sear on a pan before adding to the mix.

\subsection{Chinese Style Fish}

\paragraph{Ingredients (Marinate):}

\begin{itemize}
	\item 2 table spoon soy sauce
	\item 2 table spoon chinese wine
	\item 1 long strip of ginger, cut into pieces
	\item 2 green onion cut into pieces
\end{itemize}

\paragraph{Ingredients (Stew):}

\begin{itemize}
	\item 2 tomatoes
	\item 2 pinches of salt
	\item 1 pinch of sugar
	\item 1/2 table spoon of apple vinegar
	\item 4 table spoon soy sauce
	\item 2 table spoons chinese wine
	\item 4 garlic cloves
	\item 2 strips of ginger
	\item 1 cup of boiling water
\end{itemize}

\paragraph{Directions (Marinate):}
\begin{itemize}
	\item Mix the ingredient from the marinate together.
\end{itemize}


\paragraph{Directions (Stew):}
\begin{itemize}
	\item With the marinate ingredients, season fish. Let aside for 10min
	\item Sear both sides of fish (just the fish not the sauce)
	\item Pour all ingredients (from mariante and stew)
	\item Let it cook for 20min on low heat
\end{itemize}


\subsection{Clam Chowder}{}

\paragraph{Ingredients:}

\begin{itemize}
	\item 1 tablespoon unsalted butter
	\item 1/4 pound slab bacon or salt pork, diced
	\item 2 leeks, tops removed, halved and cleaned, then
	sliced into half moons
	\item 3 large Yukon Gold potatoes, cubed
	\item 1/2 cup dry white wine
	\item 3 sprigs thyme
	\item 1 bay leaf
	\item 2 cups cream
	\item Freshly ground black pepper to taste
\end{itemize}

\paragraph{Directions:}
\begin{itemize}
	\item Cook clams on dutch oven until they are opened -- 10-15min. If you cook already processed clams (e.g. razorclams). Cook them less time. 5-10min.
	\item Add butter and cook bacon until fat renders (bacon a bit brown)
	\item Reserve bacon bits (leave bacon fat in dutch oven)
	\item Add leeks, and let it cook for 5-10min.
	\item Add potatoes and wine and cook til wine has evaporated
	\item Add broth enough to cover the potatoes, add thyme and bay leaf
	\item Cook for 10-15min
	\item Add chopped clams, reserved bacon, cream and let it come to a simmer.
\end{itemize}

\section{Breakfast}

\subsection{Waffle}

\paragraph{Ingredients:}

\begin{itemize}
	\item 2 eggs
	\item 2 cups of all purpose flour
	\item 2 cups of milk (or nido - 1/2 cup powder + 2 cups of water)
	\item 1/2 cup of vegetable oil
	\item 1 tbl spoon sugar
	\item 4 tsp baking powder
	\item 1/4 tbl spoon salt
	\item [Optional] Mozzarella cheese and turkey (chopped)
\end{itemize}

Have some spare butter

\paragraph{Directions:}
\begin{itemize}
	\item Preheat waffle iron. Beat eggs in large bowl with hand beater until fluffy. Beat in flour, milk, vegetable oil, sugar, baking powder, salt and vanilla, just until smooth.
	\item Spray preheated waffle iron with non-stick cooking spray. Pour mix onto hot waffle iron (You can optionally interleave with the turkey and cheese). Cook until golden brown. Serve hot.
\end{itemize}

Experiment1: Add more salt (1/2). Add parmesan cheese.

\subsection{Pancakes}

\paragraph{Ingredients:}

\begin{itemize}
	\item 2 cups of all purpose flour 
	\item 2 teaspoon of baking powder 
	\item 1/2 teaspoon salt 
	\item 1 teaspoon sugar 
	\item 2 eggs 
	\item 1 1/2 cups of milk (Might need more milk)
	\item 2 tablespoons of melted and cooled butter
\end{itemize}

Have some spare butter

\paragraph{Directions:}
\begin{itemize}
	\item Mix all ingredients until there are no lumps 
	\item Turn heat on, let it get war before putting the butter 
	\item Use the 1/3 cup measurer to put pancakes into the pan 
	\item Flip after 3-5min in the pan, adjust heat accordingly if too hot
\end{itemize}

\section{Apps}

\subsection{Ceviche}

\paragraph{Ingredients:}

\begin{itemize}
	\item Freshly squeezed lemon juice
	\item Bunch of cilantro
	\item 1 tomato
	\item 2 shallots
	\item half a cucumber
	\item 1 chile serrano 
\end{itemize}

\paragraph{Directions:}
\begin{itemize}
	\item Cut
\end{itemize}

\subsection{Bruschetta}

\paragraph{Ingredients:}

\begin{itemize}
	\item 2 riped tomatoes chopped
	\item 2 minced garlic cloves
	\item basil finely chopped
	\item olive oil
\end{itemize}

\paragraph{Directions:}
\begin{itemize}
	\item Mix all ingredients.
	\item Season with salt and pepper.
	\item Put butter on both sides of the toast, put on the oven for 350F until on of the sides is brown, then put the mixture
\end{itemize}

PS1: Trick to cut the toast thin is to put on the freezer overnight
PS2: One variation is using fried garlic. But you need more. I've used 6 garlic cloves.
PS3: Another variation is using balsamic vinegar.

\subsection{Caponata (Sergio)}

\paragraph{Ingredients:}

\begin{itemize}
	\item 1 eggplant
	\item 1lb of tomatoes (3-4 tomatoes)
	\item 1 onion
	\item 5 garlic cloves
	\item 3 table spoons of chopped olives.
	\item 2 teaspoon of sherry vinegar.
	\item 1 red bell pepper
\end{itemize}

\paragraph{Directions:}
\begin{itemize}
	\item Roast eggplant (425F for 20min)
	\item With olive oil (Be generous) and on this order sear: onions, garlic, bell peppers.
	\item Once they are seared, add tomatoes and let it simmer for 30min.
	\item Once ready, add vinegar and olives.
\end{itemize}

\subsection{Eggplant in preserved olive oil}{}

\paragraph{Ingredients:}

\begin{itemize}
	\item 2 eggplant chopped in thin strips
	\item 500g salt
	\item 4 garlic cloves chopped in slices
	\item Red bell peppers chopped in thin strips
	\item 1 teaspoon oregano
	\item 2 bay leafs
	\item 1 liter of water
	\item 500ml of white vinegar
	\item 500ml of olive oil
\end{itemize}

\paragraph{Directions:}
\begin{itemize}
	\item (T-1) Cut eggplant in small thin strips (peel it first) -- same with red bell peppers
	\item (T-1) Add salt (Didn't see them adding everything)
	\item (T-1) Add a strainer and some weight on the top
	\item (T-1) Leave in the fridge overnight
	\item (T) Wash the eggplant/peppers in 1:2 between vinegar and water (500ml of vinegar / 1000ml of water)
	\item (T) Start layering ingredients (garlic then eggplant and pepper mixture)
	\item (T) Add olive oil til the top
	\item (T) Let it sit for 2 weeks
\end{itemize}

\subsection{Caponata in Oven}
\paragraph{Ingredients:}

\begin{itemize}
	\item 10 roma tomatoes roughly chopped
	\item 1 onion
	\item 10 green olives
	\item 8 garlic cloves (whole)
\end{itemize}

\paragraph{Directions:}
\begin{itemize}
	\item Mix everything with olive oil, salt and ground pepper.
	\item Turn oven to 350F and leave for a 2-3hours. Mix every 30mins.
\end{itemize}

\subsection{Garlic bread}

\paragraph{Ingredients:}

\begin{itemize}
	\item 2 cups - Cream cheese (Whipped is better)
	\item 1/2 cup - Mayonese 
	\item Garlic (5 tablespoons)
	\item Parmesan
	\item Sharp Sheddar
	\item Pepper
	\item Oregano
	\item Bread (Take and bake are better)
\end{itemize}

\paragraph{Directions:}
\begin{itemize}
	\item Mix all ingredients
	\item Put oven on 425F for 10-15'
\end{itemize}

\subsection{Paprika Toasts}

\paragraph{Ingredients:}

\begin{itemize}
	\item Smoked Paprika
	\item Olive oil
	\item Cream cheese
	\item Cheese (Mozzarella, Parmesian)
	\item Salt
	\item Toasts
\end{itemize}

\paragraph{Directions:}
\begin{itemize}
	\item Mix all ingredients (You can also use mayo if you want)
	\item Turn oven on 350F and once hot add toasts before so that its slightly toasted before you add the spread - 5' each side
	\item Remove from oven put spread
	\item Put for more 5' in the oven
\end{itemize}


\section{Spanish Dishes}


\subsection{Bravas Sauce}

\paragraph{Ingredients:}

\begin{itemize}
	\item 1/3 cup of olive oil
	\item 1/2 Tbsp. of pimentón picante hot smoked paprika
	\item 1 1/2 Tbsp. of pimentón dulce sweet smoked paprika
	\item 1 –2 Tbsp. of flour
	\item 1 cup of chicken broth or vegetable broth, for a vegetarian version
	\item Heavy cream
	\item Salt to taste
\end{itemize}

\paragraph{Directions:}
\begin{itemize}
	\item Heat the olive oil in a small saucepan over medium heat.
	\item Add the pimentón dulce and pimentón picante and stir until combined.
	\item Add 1 tablespoon of flour and stir until combined. Keep stirring for about a minute, to toast the flour slightly.
	\item Over a medium-low heat, add the broth very gradually, stirring constantly. (This is similar to how you'd make a cream sauce. The flour will absorb the liquid and leave you with a delicious sauce.)
	\item The sauce should start to thicken as you incorporate the broth; add more flour only if necessary to achieve the right consistency (it should be velvety and smooth, but not so thick that it holds its shape alone).
	\item Reduce to low heat and simmer for 3-5 minutes, stirring occasionally.
	\item Season with salt to taste.
	\item Drizzle over some fried potatoes and enjoy!
\end{itemize}

\subsection{Patatas Bravas}

\paragraph{Ingredients:}

\begin{itemize}
	\item 4 medium potatoes
	\item Olive oil for frying, about 1-2 cups
	\item Salt
	\item 1/2 cup of homemade bravas sauce
\end{itemize}

\paragraph{Directions:}
\begin{itemize}
	\item Peel the potatoes, rinse thoroughly, and dry with a paper towel.
	\item Cut the potatoes into bite-size chunks.
	\item Heat the olive oil in a large skillet over medium heat.
	\item Add the potatoes and adjust the heat to the lowest setting, allowing them to pre-cook for a few minutes.
	\item Remove the potatoes and let them cool in the fridge for a few more minutes.
	\item Turn the heat up to high and add the potatoes back into the pan.
	\item Fry until crispy and golden.
	\item Transfer the potatoes to a plate lined with paper towels to cool, and sprinkle with salt to taste.
	\item To serve, drizzle the bravas sauce over the potatoes.
	\item Dig in—no forks necessary!
\end{itemize}

\subsection{Croquetas de Jamon}

\paragraph{Ingredients:}

\begin{itemize}
	\item 4 tbsp unsalted butter (60 g)
	\item 1/4 cup olive oil (60 ml)
	\item 1 scant cup flour (just under one cup — 120 g)
	\item 1 medium onion very finely diced
	\item 1/4 gallon whole milk at room temperature (1 liter)
	\item 1 pinch nutmeg
	\item 1/2 pound jamón serrano diced into small pieces (225 g)
	\item flour for breading
	\item 2 beaten eggs
	\item bread crumbs for breading (try Panko for non-traditional extra crispy croquettes!)
\end{itemize}

\paragraph{Directions:}
\begin{itemize}
	\item Melt the butter and warm the oil in a heavy pan over medium high heat.
	\item Add the diced onion and sauté for a few minutes, until it just starts to color.
	\item Add a pinch of salt and the nutmeg. Don't add too much salt as the Serrano ham is already salty.
	\item Add the diced ham and sauté for 30 seconds more.
	\item Add the flour and stir continuously, until the flour turns a light brown color. You must not stop stirring or the flour will burn!
	\item When the flour changes color, add the milk little by little, always stirring until you incorporate the entire amount. It should take about 15-20 minutes to add it all.
	\item Turn off the heat and let the dough cool to room temperature.
	\item Butter the sides of a large bowl and place the croquette dough inside, covered directly with plastic wrap. Refrigerate a minimum of 4 hours, but preferably overnight.
	\item To make the ham croquettes, shape them into little logs (or use a pastry sleeve if you have one.)
	\item Next, while heating a pan full of olive oil on the stove, pass the croquettes through the three step breading process. First, cover them in flour, then in egg, and, finally, in the breadcrumbs.
	\item Fry the ham croquettes in the hot oil for about five minutes (making sure to turn halfway so they brown evenly) and then let them cool for a few minutes before enjoying!
\end{itemize}


\subsection{Paella valenciana}

\paragraph{Ingredients:}

\begin{itemize}
	\item 500g Rabbit
	\item 500g Chicken
	\item 60 cl Extra virgin olive oil
	\item 1 ripe tomato
	\item 200g of green beans
	\item 200g g of garrofón (lima bean)
	\item 500g Bomba rice (or any other short-grain rice if you can’t get it)
	\item 1.5 liters chicken or vegetable stock.
	\item Sprig of fresh rosemary
	\item Saffron from La Mancha (if you don’t have any don’t worry, you can use a Paella Seasoning like this one )
	\item Salt
\end{itemize}

\paragraph{Directions:}
\begin{itemize}
	\item Heat up the paella, add the oil and when it gets quite warm, add the meat (chicken and rabbit, cut into small pieces).
	\item Sauté it over low heat until the meat is sealed and golden.
	\item The next step is to add the tomatoes (peeled and ground) and vegetables (lima and green beans), maintaining the same heat.
	\item Once everything is well fried, add the stock, a sprig of rosemary and heat everything up.
	\item Just when it begins to boil, add the rice, the snails, salt and saffron and remove the rosemary. At this moment the fire needs to be at its maximum.
	\item When the rice is cooking for about 10 minutes, decrease the heat gradually for at least another ten minutes.
	\item Once the paella is done and all the liquid has evaporated, let it stand for a couple of minutes to let it form the socarrat or socarradet (light crust of rice on the bottom of the pan) and then it’s ready.
\end{itemize}

\section{Weekend Meals}

\subsection{Burger}

\paragraph{Ingredients:}

\begin{itemize}
	\item 1 garlic clove minced
	\item 1 pound of 80-85\% meat
	\item 1-2 oz of bacon chopped in small pieces
	\item 2 egg yolk
	\item 4 burger buns
	\item 4 table spoons Mayonnaise
	\item 1/4 of onion seared (add some salt and sugar)
	\item 1 tablespoon of worcestershire sauce
	\item 4 cheese slices
\end{itemize}

\paragraph{Directions (For aioli):}
\begin{itemize}
	\item Mix Mayonnaise, worcestershire sauce and seared onion. You want to blend that.
\end{itemize}

\paragraph{Directions:}
\begin{itemize}
	\item Fry the bacon, reserve meat and fat
	\item Wait bacon to cool
	\item Mix egg yolk, meat, bacon and garlic and make some nice paddies.
	\item Fry the burger paddies on the bacon fat
	\item Once its close to done (125F) add the cheese and cover for the cheese to melt.
	\item Toast the buns lightly
	\item Assemble the dish adding aioli
\end{itemize}

\subsection{Chicken Fricassee}

\paragraph{Ingredients:}

\begin{itemize}
	\item 1 Onion
	\item 2lb of chicken
	\item 1 tablespoon of butter
	\item Batata palha
	\item 2 Requeijao
\end{itemize}

\paragraph{Directions:}
\begin{itemize}
	\item Blend onion on a blender (You might need to add a bit of olive oil in order to blend)
	\item Add chicken to skillet, sear all sides
	\item Then add onion mixture
	\item Let it cook for a little bit.
	\item Shred chicken
	\item Add in onion sauce until the consistency seems correct
	\item Put chicken in a Pyrex
	\item Then add the requeijao
	\item Then add the batata palha
\end{itemize}

\subsection{Feijao tropeiro}

\paragraph{Ingredients:}

\begin{itemize}
	\item 1 bunch of collar greens, cut in thin slices
	\item 3 cups of bacon, in small pieces
	\item 2 cups of uncooked pinto beans
	\item 6 smoked pork sausages (used calabresa)
	\item 6 scrambled eggs
	\item 1 full bulb of garlic
	\item 1 large onion
	\item 2 cups of cassava flour
	\item 1/2 cup of green onion.
\end{itemize}

\paragraph{Directions:}
\begin{itemize}
	\item Cook all bacon, reserve meat and oil
	\item Cook all sausage, reserve	
	\item Cook all green onion, reserve
	\item Cook all onion + garlic, reserve
	\item Cook in the pressure cooker the pinto beans for 10min
	\item Toast the cassava flour. This requires patience so that it doesn't burn.
	\item Make the scrambled eggs. Ideally all using the same pot.
	\item Just before mixing all ingredients, cook the collard greens.
\end{itemize}

\subsection{File Mignon}

\paragraph{Ingredients (Marinate):}
\begin{itemize}
	\item 2lb of Beef tenderloin (Make sure to remove fat otherwise sauce gets really greasy)
	\item 12 garlic cloves (~1 package of garlic)
	\item 1/2 of chipped onion
	\item 1 tomato without seeds (I remove the inner part entirely)
	\item 2-3 tablespoons of red wine vinegar
	\item 1-2 teaspoons of cumin
	\item Salt and pepper
\end{itemize}

\paragraph{Ingredients (Searing):}
\begin{itemize}
	\item 2 tablespoons of margarine (I've used Country Rock)
	\item 2 tablespoons of oil
	\item 4oz of red wine
	\item 1 tablespoon of tomato extract
	\item 1 cup of water
	\item 2 cans of while mushrooms
\end{itemize}

\paragraph{Directions (Marinate):}
\begin{itemize}
	\item Poke small wholes in the tenderloin for the marinate to penetrate.
\end{itemize}

\paragraph{Directions (Searing):}
\begin{itemize}
	\item In an aluminium pan, put margarine and oil. Let it get super hot.
	\item Reserve the veggies. Sear the two sides of the tenderloin.
	\item Reserve the tenderloin and let the oil get hot again (Repeat this step as needed).
	\item Brush the tenderloin in the pan.
	\item Sear all sides of the tenderloin.
	\item Reserve the tenderloin.
	\item Put the veggies in the pan, let it sear. Sometimes I put a bit of water if the pan is burning too much.
	\item Once veggies are very soft, put wine and let it cook a bit.
	\item Add Tomato extract and water, let it cook for 5-10min.
	\item Taste and see if it needs salt.
	\item Once ready, put veggies on a strainer and using a spoon apply some force so that the veggies go over the strainer so that the sauce thickens.
	\item Add the mushrooms and tenderloin back in the pan.
	\item Done.
\end{itemize}

\subsection{Camarao ao Tornedor}

\paragraph{Ingredients (Shrimp):}
\begin{itemize}
	\item 3lb of shrimp
	\item 1 lime
	\item 3 table spoons of olive
\end{itemize}

\paragraph{Ingredients (Sauce):}
\begin{itemize}
	\item 200ml of flour
	\item 1 cup of powdered milk (nido)
	\item Water (TBD)
\end{itemize}

\paragraph{Ingredients (Searing):}
\begin{itemize}
	\item 3/4 chopped onion
	\item 1 tablespoon oil
	\item 4 tablespoons of margarine (Country rock)
	\item 1 tablespoon of tomato extract
	\item 4 oz white while
	\item 2 cans of creme de leche (~200g each)
	\item 1 cup of shredded parmesan cheese
\end{itemize}

\paragraph{Directions (Shrimp):}
\begin{itemize}
	\item Squeeze lime with water in a container
	\item Remove skins and organ from shrimp and add to the container with mixture.
	\item Let the shrimp stay in the water with lime for 20min (Usually its just the time of prepping the shrimp)
	\item Wash the shrimp to remove the lime
	\item Put the shrimp in a saucepan with water
	\item Turn off when the shrimp turn lightly pink (The water doesn't even boil)
	\item Throw away the water, put shrimp with olive oil, salt and pepper in a container.
\end{itemize}

\paragraph{Directions (Sauce):}
\begin{itemize}
	\item Mix flour, powdered milk and mix well. Add water.
\end{itemize}

\paragraph{Directions (Searing):}
\begin{itemize}
	\item Put margerine and oil in a saucepan
	\item Add onion, let the onion get trans-lucid.
	\item Add the shrimp, add wine, white sauce, tomato extract and parmesan cheese.
	\item No need to boil (its still going to the oven), Add the creme de leche.
	\item Test the salt
	\item Put in a dish for going to the oven, add parmesan on the top.
	\item 425F for 30min.
\end{itemize}

\subsection{Boef Bourguignon}

\paragraph{Ingredients:}

\begin{itemize}
	\item 4 ounces of bacon, sliced crosswise into thin strips (1/4 inch by 1 inch pieces)
	\item 2 pounds of boneless beef chuck, trimmed of all fat and cut into 1 inch cubes
	\item 1 carrot, peeled and sliced into 1/4 inch think rounds
	\item 1 large yellow onion, medium diced
	\item 2 tablespoon unbleached all-purpose flour
	\item 2 cups full-bodied, red wine like chianti or merlot
	\item 2-3 cups of low sodium beef stock
	\item 1 tbl spoon tomato paste
	\item 3 large garlic cloves, finely minced
	\item 3 sprigs fresh thyme, tied with butcher's twine
	\item 2 bay leaves
	\item 2 tbl spoon of butter, room temperature
	\item 3 tbl spoons chopped flat-leaf parsley
	\item 1/6 cup sherry vinegar
\end{itemize}

\paragraph{Directions:}
\begin{itemize}
\item Preheat oven to 325
\item In a large dutch oven over medium heat, saute bacon until lightly browned. About 5 minutes. Transfer bacon into a medium bowl and reserve fat in the pan.
\item Dry the beef thoroughly with paper towel and season with salt and pepper. Return dutch oven to medium-high heat. When fat is shimmering, add the beef and sear all sides. Transfer seared beef to bowl with bacon.
\item Reduce heat to medium and add the carrots and onions to the pan. Saute until lightly browned, about 8 minutes.
\item Return the beef and bacon to the dutch oven with the onion and carrots; season lightly with salt and pepper. Sprinkle flour over and toss to a lightly coat.
\item Slowly stir in wine and add enough stock to just cover the meat.
\item Stir in tomato paste, garlic, thyme sprigs and bay leaf.
\item Bring to a simmer over medium heat. 
\item Cover the pot and place in the preheated oven for 1 1/2 to 2 hours.
\item The meat is done when a fork pierces it easily.
\item Separate the solids from the liquid with a sieve.
\item Discard thyme sprigs and bay leaf.
\item Put liquid in a saucepan with the sherry vinegar and let is simmer until liquid is required by half.
\item Add roasted onions and mushroom to the dutch oven.
\item Skim fat off the surface of the braising liquid with a laddle and discard fat.
\item Season with salt and pepper. Maybe adding a bit of sugar to balance the acidity (Used 3 teaspoons of coconut sugar).
\item Pour mixture to dutch oven and stir to combine.
\end{itemize}

\paragraph{Ingredients (Onions)}
\begin{itemize}
	\item 15 pearl onions
	\item 8 ounces cremini or button mushrooms, trimmed and quartered
	\item 1/6 cup sherry vinegar
	\item butter
\end{itemize}


\paragraph{Directions (Onions)}
\begin{itemize}
\item While the beef is cooking, blanch the pearl onions in rapidly boiling salted water until tender, about 4 minutes. 
\item Drain and let cool, then trim ends and peel. Set aside
\item To a large ovenproof skillet over medium-high heat, add butter.
\item Add onions and mushrooms and stir to coat vegetables with the melted butter.
\item Drizzle with sherry vinegar and season generously with salt and pepper.
\item Place skillet in a oven, stirrring occasionally, until onions and mushrooms turn golden brown and most of the liquid have evaporated. Around 20-25minutes.
\item Remove from oven and set aside.
\end{itemize}

\subsection{Steakhouse Steak}

\paragraph{Ingredients:}

\begin{itemize}
	\item 2 good steaks (Ribeye or New york - look for dry aged prime beef)
	\item 5 cloves of garlic
\end{itemize}

\paragraph{Directions:}
\begin{itemize}
	\item Turn oven at 450F
	\item Pat steak with a paper towel to remove any excess of moisture
	\item Season with kosher salt, group pepper and maybe coriander also.
	\item Let it sit for 10'
	\item Heat skillet
	\item Just before adding the steak add butter and a touch of olive oil.
	\item (If you bought a cheap steak you might want to add garlic)
	\item Sear each side of the steak (~2' each)
	\item Put the steak on the oven for 10-15' (For medium)
	\item Let it sit for 2-3' before serving.
\end{itemize}

\subsection{Torta de Ana}

\paragraph{Ingredients (For dough):}

\begin{itemize}
	\item 6 cups of flour
	\item 2 egg yolks
	\item 2 table spoons of Parmigiano cheese
	\item 7 tbl spoon of mayonnaise
	\item 3 blocks of butter - or 1 2/3 cup of margarine
\end{itemize}

\paragraph{Ingredients (For chicken):}

\begin{itemize}
	\item 2 chicken breasts
	\item lime
	\item creme de leche
	\item 1 onion
	\item 1 tomato
	\item 1/2 bell pepper
	\item scallion
	\item 3 garlic cloves
	\item 2 knoors (one for each breast)
	\item cumin
	\item cilantro
	\item tomato paste
	\item olive
	\item green peas
\end{itemize}

\paragraph{Directions:}
\begin{itemize}
	\item Cut chicken in small cubes put in a container with water and lime. Let it sit for a couple of minutes.
	\item Sear the vegetables with the knoor (onion, tomatoes, peppers, scallion and garlic).
	\item After the garlic is fragrant add the chicken, cumin, cilantro and tomato paste. 
	\item Let it simmer for 30'
	\item While chicken cooks, make the dough (mix all ingredients).
	\item When the chicken is ready, shred the chicken and put the juices on the side.
	\item Mix chicken, creme de leche and sauce to taste. 
	\item Add olives and green peas
	\item Wait for chicken to cool
	\item Put dough in a pie container (use the plastic wrap technique for building the top)
	\item Put chicken and cover the top
	\item Use egg yolks and Parmesan to decorate the top
	\item Turn oven to 325F and put pie for 30'
	\item Increase heat to 425F for finish
\end{itemize}

\paragraph{Notes:}

\begin{itemize}
\item You can use the dough if any left to make some snacks. Add some smoked paprika with Parmigiano cheese.
\item If any sauce from the chicken is left, its an excellent base for rice.
\end{itemize}

\subsection{Slow Cooked BBQ Ribs}

\paragraph{Ingredients (for rub):}

\begin{itemize}
\item 1/2 cup brown sugar
\item 2 tablespoons paprika
\item 1 tablespoon smoked paprika
\item 1 tablespoon black pepper
\item 1/2 tablespoon kosher salt
\item 1 tablespoon garlic powder
\item 1 tablespoon onion powder
\item 1 teaspoon mustard powder
\end{itemize}

\paragraph{Ingredients:}

\begin{itemize}
\item Jack Daniels BBQ Sauce 
\item 1 Rack of Ribs 
\item Ribs Seasoning (Safeway would do it)
\end{itemize}

\paragraph{Directions:}
\begin{itemize}
\item Put seasoning in both sides of ribs 
\item Wrap the ribs in aluminum foil 
\item Turn oven on 325F and let it cook for 3-4hrs 
\item Remove aluminum foil and put BBQ sauce 
\item Turn the heat up to 425F and leave for more 10min 
\item Cut and serve
\end{itemize}

\subsection{Peixe ao coco}

\paragraph{Ingredients:}

\begin{itemize}
	\item Azeite de dende (3 colheres de sopa)
	\item 2 lata de leite de coco (coconut thai)
	\item 1 pimentao vermelho
	\item 1 pimentao laranja
	\item 1 pimentao verde (pequeno)
	\item 1 1/2 cebola
	\item 2.5lb (1.1kg) posta de peixe (tilapia)
	\item 2-3 tomates
	\item coentro
\end{itemize}

\paragraph{Directions:}
\begin{itemize}
	\item Deixe o peixe descansando no limao, sal e pimenta
	\item Refogue o pimentao e a cebola no dende
	\item Quando o pimentao e cebola ficarem macios jogue o tomate e coentro
	\item Preaqueca o forno a 400F
	\item Coloque o leite de coco, deixe aquecer mas n deixe ferver. 
	\item Tempere com sal a gosto
	\item Num refratario coloque o peixe e a mistura anterior
	\item 15'no forno eh o suficiente.
\end{itemize} 

\subsection{Steak au Poivre}

\paragraph{Ingredients:}

\begin{itemize}
	\item 4-6 1 1/2 inch thick steaks
	\item Kosher salt
	\item 4 table spoons of whole peppercorn
	\item 1 tablespoon unsalted butter
	\item 1 tsp of olive oil
	\item 1/4 cup of cognac (Original Elton Brown recipe calls for 1/4)
	\item 1 cup of heavy cream
\end{itemize}

\paragraph{Directions:}
\begin{itemize}
	\item Remove the steaks from the refrigerator for at least 30 minutes and up to 1 hour prior to cooking. Sprinkle all sides with salt. 
	\item Coarsely crush the peppercorns with a mortar and pestle, the bottom of a cast iron skillet, or using a mallet and pie pan. Spread the peppercorns evenly onto a plate. Press the fillets, on both sides, into the pepper until it coats the surface. Set aside
	\item In a medium skillet over medium heat, melt the butter and olive oil. As soon as the butter and oil begin to turn golden and smoke, gently place the steaks in the pan. For medium-rare, cook for 2-3 minutes on each side. Once done, remove the steaks to a plate, tent with foil and set aside. Pour off the excess fat but do not wipe or scrape the pan clean. 
	\item Off of the heat, add 1/3 cup Cognac to the pan and carefully ignite the alcohol with a long match or firestick. Gently shake pan until the flames die. Return the pan to medium heat and add the cream. Bring the mixture to a boil and whisk until the sauce coats the back of a spoon, approximately 5 to 6 minutes. Add the teaspoon of Cognac and season, to taste, with salt. Add the steaks back to the pan, spoon the sauce over, and serve.
\end{itemize}

\begin{description}
	\item[Note1] If the sauce is too bitter add a bit of sugar.
	\item[Note2] If you need to cook many batches of the recipe use many pans so that the bits in the pan don't get cooked too much.
\end{description}

\subsection{Parisian Gnocchi}

\paragraph{Ingredients (For dough):}

\begin{itemize}
	\item 1 7/8 cup of flour
	\item 1 1/2 cup of water
	\item 6oz of unsalted butter
	\item 3/4 tsp kosher salt
	\item 1 tablespoon dijon mustard
	\item 3/4 cup of grated parmesan cheese
	\item 4 eggs
	\item 2 tbl spoon parsley leaves
\end{itemize}

\paragraph{Ingredients (For stir frying the gnocci):}

\begin{itemize}
	\item butter
	\item olive oil
	\item 2 table spoons chives
	\item 6 pureed garlic cloves
\end{itemize}


\paragraph{Directions:}
\begin{itemize}
	\item Bring water, butter, and salt to a boil in a medium saucepan over high heat. Add flour all at once and stir with a wooden spoon until a smooth dough forms. Reduce heat to medium-low and continue to stir, beating dough forcefully and rapidly to prevent it from sticking to the pot. Continue cooking until dough pulls away from sides of pot leaving a thin layer and steams slightly.
	\item Transfer hot dough to the bowl of a stand mixer fitted with a paddle attachment. Add mustard and cheese and beat on medium-low speed. Add eggs one at a time, allowing dough to fully incorporate egg before adding the next one. When final egg has been added, add herbs and beat to combine. Transfer mixture to a gallon-sized zipper-lock bag or a pastry bag fitted with a 1/2-inch tip.
	\item Let mixture rest 15 to 25 minutes at room temperature. Meanwhile, bring a large pot of salted water to a simmer and have a rimmed baking sheet. If using a zipper-lock bag, cut off a 1/2-inch opening in one corner. Holding the bag over the boiling water, squeeze the mixture out of the bag, cutting it off with a paring knife into 1-inch lengths and letting them fall directly into the simmering water. Continue cutting off as many as you can in one minute, then stop.
	\item When all gnocchi have floated to the top, continue cooking until gnocchi are fully cooked to the center, about 3 minutes longer. 
	\item In a non-stick pan, add a bit of butter and olive oil, add gnocchi and garlic until the gnocci is golden brown. Do in batches.
\end{itemize}

\subsection{Asian Glazed Salmon in Foil}

\paragraph{Ingredients:}
\begin{itemize}
  \item 1/4 cup soy sauce low sodium
  \item 3 tbsp maple syrup
  \item 2 tbsp sweet chili sauce
  \item 2 tbsp lime juice freshly squeezed
  \item 1 tsp fresh ginger minced
  \item 3 cloves garlic minced
  \item salt and pepper to taste
  \item 1-2 lb salmon fillet no skin
  \item 1 tbsp sesame seeds
  \item 2 green onions chopped
\end{itemize}

\paragraph{Directions:}
\begin{itemize}
  \item Preheat your oven to 375 F degrees. Place 2 foil sheets on a baking sheet. I used 2 foil sheets because I
want to make sure there’s enough foil to wrap around the salmon.
  \item In a small bowl whisk together the soy sauce, maple syrup, sweet chili sauce, lime juice, fresh ginger,
garlic, salt and pepper. The salt can probably be omitted since the soy sauce should have plenty of
sodium.
  \item Place the salmon fillet over the foil then pour about half the marinade over and brush if necessary to
cover the entire surface of the fish. Turn the fish over and pour the rest of the marinade. Also brush the
marinade making sure that the entire fish is brushed with marinade.
  \item Fold the edges of the aluminum foil over the salmon so that it is fully wrapped. Bake for about 20 to
25 minutes, depending on the size of your salmon. My salmon was quite a big piece, so I baked it for 25
minutes. You may also turn the broiler on and broil for about 2 to 3 minutes at the end if preferred.
  \item Open the foil and garnish with sesame seeds and chopped green onions.
\end{itemize}

\subsection{Teriyaki Salmon}{}

\paragraph{Ingredients:}
\begin{itemize}
	\item 3 tablespoons of teriyaki sauce (might use more)
	\item 3 tablespoons hoisin
	\item 3 tablespoons soy sauce
	\item 1 tablespoon white vinegar
	\item 1 tablespoon of sesame oil
	\item 1/3 cup brown sugar
	\item 3 garlic cloves minced
	\item 2 teaspoon of freshly grated ginger
\end{itemize}

\paragraph{Directions:}
\begin{itemize}
	\item Season salmon with above mixture
	\item You can cook either in oven (~@400F) or in the frying pan
\end{itemize}

\subsection{Teriyaki Sauce}

\paragraph{Ingredients:}

\begin{itemize}
	\item 1/2 cup low sodium soy sauce
	\item 1/4 cup brown sugar
	\item 1 1/2 teaspoon of fresh ginger
	\item 1 teaspoon of garlic
	\item 1 tablespoon honey
	\item 1 teaspoon sesame oil
	\item 3 tablespoons mirin
	\item 1/4 cup of water mixed with 3 teaspoons cornstarch
\end{itemize}

\paragraph{Directions:}
\begin{itemize}
	\item Mix everything and leave in a small sauce pan in low heat until it thickens a bit
\end{itemize}

\section{Pressure Cook}

\subsection{Beef Ragu (Tia Angela)}

\paragraph{Ingredients:}

\begin{itemize}
	\item 1 good cut of meat (I actually used last time Skirt Steak)
	\item 1 Bottle of 12oz of Guinness Beer
	\item Olive oil
	\item 1 jar of tomato sauce (Used Barilla tradition)
	\item 1 Packet of Onion Soup (Used lipton)
	\item 1 tablespoon of sugar
\end{itemize}

\paragraph{Directions:}
\begin{itemize}
	\item In the pressure cooker add olive oil and tablespoon of sugar in the pan (Original recipe uses teaspoon of sugar)
	\item Sear all sides of the steak
	\item Throw remainder of ingredients (Tomato sauce, Beer and Onion Soup), stir until mixture is uniform.
	\item Leave in the pressure cooker for 35-40'
\end{itemize}

\paragraph{Note: The chicken version of this dish you replace beer for 1/2 cup of white wine, and 1 cup of chicken stock and put grated onion instead of the onion soup}


\subsection{Bife de Molho Luzinete}

\paragraph{Ingredients:}

\begin{itemize}
	\item Beef cut into thin steaks
	\item 1 onion
	\item 1 tomato
	\item bunch of cilantro
	\item cumin
	\item 2 garlic cloves
	\item 1 knoor block
	\item tomato sauce (1 1/2 cup)
	\item 1/2 cup of water
	\item 1 cup of mozzarella cheese
\end{itemize}

\paragraph{Directions:}
\begin{itemize}
	\item Season the beef with knoor, cumin and garlic
	\item On a sauté pan sear the beef both sides
	\item Add the (onion, tomato, cilantro, tomato sauce and water)
	\item Put in the pressure cook for 15'
	\item If require cook without pressure for more 10'
	\item Turn the oven on 425F
	\item Put beef on a bowl, add mozzarella cheese and put on the oven for 15' or until cheese is melt.
\end{itemize}

\subsection{Savory Beef Tips and Gravy}

\paragraph{Ingredients:}
\begin{itemize}
  \item 1 cup finely chopped onion
  \item 2 tomatoes chopped
  \item 4 garlic cloves minced
  \item 1 1/2 lb beef chuck cut into pieces
  \item 1/4 cup flour
  \item kosher salt
  \item fresh ground pepper
  \item 1/4 cup red wine
  \item 1 cup beef stock
  \item 2 table spoons Worcestershire Sauce	
\end{itemize}

\paragraph{Directions:}
\begin{itemize}
  \item Season beef with salt and pepper and use flour to coat.
  \item Add oil to large skillet, sear all sides of the beef, 5-10’. Reserve.
  \item Add a little bit more of oil and add onions. Wait until onion gets translucent. 3’. Add garlic until its
fragrant. Around 1’.
  \item Deglaze pan with wine, cook on medium/low heat for 1-2’, then add stock, worcestershire sauce and
tomatoes.
  \item Cook under pressure for 20’.
\end{itemize}

\subsection{Pork Cassoulet}

\paragraph{Ingredients (Beans):}
\begin{itemize}
	\item 1 lb of white beans
	\item 1 carrot, chopped
	\item 1 bouquet garni (thyme, 2 bay leaves, parsley)
	\item salt to taste
\end{itemize}

\paragraph{Directions (Beans):}
\begin{itemize}
	\item Put the beans on water for 3 hours.
	\item Drain the water, then mix all ingredients and cook under pressure for 20'
\end{itemize}

\paragraph{Ingredients (Main):}
\begin{itemize}
	\item 1-2 lb of duck confit	
	\item 2 lb of pork shoulder
	\item 4 garlic cloves minced
	\item 5 oz of bacon
	\item 1 smoked pork hock
	\item 4-6 slices of bacon or 6oz, chopped
	\item 1 smoked sausage, sliced (cooked)
	\item 1 large onion, chopped
	\item 2 tomatoes, chopped
	\item 1 tbl spoon tomato paste
	\item 4 garlic cloves, minced
	\item 1 table spoon of tomato paste
\end{itemize}

\paragraph{Directions (Main):}
\begin{itemize}
	\item Turn oven on 375F, season \textbf{duck} with salt and olive oil and let it bake for around 40'.
	\item Sear the \textbf{bacon} first, until browned. Reserve.
	\item With the bacon fat, sear the \textbf{pork shoulder}. Reserve
	\item With the bacon fat, sear lightly the \textbf{sausage}. Reserve
	\item With the bacon fat (maybe add some oil if the fat is gone), add \textbf{onion}
	\item Once onion is translucent add \textbf{garlic} until fragrant.
	\item Then Deglaze the pan with a bit of water, then add tomatoes, tomato paste, pork hock all other cooked ingredients (beans and meat)
	\item Cook in the oven (with a dutch oven) on 325F for 1-2hr.
\end{itemize}

\paragraph{Ingredients (Bake):}
\begin{itemize}
	\item Breadcrumbs
\end{itemize}

\paragraph{Directions (Bake):}
\begin{itemize}
	\item Once stew is fully cooked, put on a pyrex baking dish, cover with breadcrumbs
	\item Bake on 400F for 15'
\end{itemize}

\subsection{Pressure Cook Pork (Receita Vaco)}

\paragraph{Ingredients:}
\begin{itemize}
	\item	1-2 lb of pork loin cut into cubes
	\item 2 large russet potatoes (1 2/3lb) cut into large chunks
	\item 8 large cremini mushrooms (1/2 lb) roughly chopped
	\item 2 carrots (1/2 lb) cut into large chunks
	\item 2 tablespoons of butter
	\item 2 tablespoons of light soy sauce
	\item 1 tablespoon of olive oil
	\item 4 cloves garlic, minced
	\item 2 bay leaves
	\item 1 table spoon of balsamic vinegar
	\item 2 cup chicken stock
	\item salt and pepper to taste
	\item (Optional) Corn starch or flour for thickening.
\end{itemize}

\paragraph{Directions:}
\begin{itemize}
	\item Season pork generously with salt and pepper
	\item Add olive oil and sear all sides of the pork until they are brown. Reserve the pork.
	\item Add butter, saute the mushrooms, season with salt and pepper. Stir until mushrooms are slightly
	crisp and browned. This step should take around 7-12 minutes.
	\item Add garlic, carrots and bay leaves. Saute for 2 minutes.
	\item De-glaze with a dash of balsamic vinegar add chicken stock and soy sauce.
	\item Add pork, potatoes and cook under pressure for 15’.
	\item If necessary add a bit of cornstarch for thickening the gravy (You might want to mix separate from the
	main sauce).	
\end{itemize}

\section{Ground Beef Based}

\subsection{Ground Beef}

\paragraph{Ingredients:}

\begin{itemize}
	\item 1 onion 
	\item 1 tomato 
	\item 1/4 red pepper 
	\item cilantro 
	\item cumin 
	\item salt 
	\item pepper 
	\item ground beef (Use low fat beef) 
	\item knoor 
	\item tomato paste (1 can)
\end{itemize}

\paragraph{Directions:}
\begin{itemize}
	\item Chop onion, tomatoes, red pepper, cilantro. On a big sauce, put veggies with some olive oil. Stir for a couple of min until smells. Mix in knoor, beef, cilantro, cumin 
	\item Mix in tomato paste 
	\item Season with salt and pepper 
	\item Let it simmer for a good 30min. Be careful for bottom of pan not burn
\end{itemize}

\subsection{Husband-Fattening Casserole}

\paragraph{Ingredients:}

\begin{itemize}
	\item Ground Beef Recipe
	\item Mashed Potatoes Recipe
	\item Mozzarella cheese
\end{itemize}

\paragraph{Directions:}
\begin{itemize}
	\item Turn oven to 425F
	\item In a container put the ground beef and top with the mashed potatoes.
	\item Top over with cheese
	\item Wait for 20-30min until cheese melted and formed a crust
\end{itemize}

\subsection{Hamburgers}

\paragraph{Ingredients:}

\begin{itemize}
	\item Good quality ground beef
	\item 1/2 onion
	\item 2-3 cloves of garlic
	\item garlic salt and onion salt or
	\item 1 table spoon of finaly chopped bacon
	\item salt
	\item pepper
	\item olive oil
	\item cheese
\end{itemize}

\paragraph{Directions:}
\begin{itemize}
	\item On a sautee pan add onion for 2-3' then garlic
	\item When onion is caramelized, turn off the heat. Reserve.
	\item On a bowl, mix ground beef, [onion salt, garlic salt] or [bacon], salt and pepper
	\item Make the patties, using the same pan add 1 tablespoon of olive oil and 1/2 part of butter. Sear both sides of steak than cover so it can cook.
	\item While burger is cooking, but buns into the oven at 400 for 5min.
	\item Wait for 5', add cheese on top, cover again for 1'
\end{itemize} 

\subsection{Molho burger}

\paragraph{Ingredients:}
\begin{itemize}
	\item 1/2 onion chopped
	\item 2 pinches of salt
	\item 1 pinch of sugar
	\item 2 table spoons of Worcestershire Sauce
	\item 1/2 cup mayo
\end{itemize}

\paragraph{Directions:}
\begin{itemize}
	\item In a pan, add oil and fry onion
	\item Add the salt and sugar
	\item Once onion is caramelized, add Worcestershire Sauce
	\item Turn heat off, let it cool and blend in mixer, then mix mayo
\end{itemize} 

\subsection{Meatballs}

\paragraph{Ingredients (Meatballs):}

\begin{itemize}
	\item 1 pound lean ground beef 
	\item 1/2 cup fresh bread crumbs 
	\item 1 tablespoon grated Parmesan cheese 
	\item 1 teaspoon of onion powder
	\item 1 teaspoon of garlic powder
	\item puree of 4 garlic cloves 
\end{itemize}

\paragraph{Ingredients (Sauce):}
\begin{itemize}
	\item 2 cups of tomato sauce (Cento brand preferably)
	\item some spice: either italian seasoning or chopped parsley
\end{itemize}

\paragraph{Directions (Meatballs):}
\begin{itemize}
	\item Mix all the meatball ingredients and make small meatballs.
	\item Add a bit of olive oil to a pan, sear the meatballs in high heat both sides, then add the tomato sauce and spice, lower the heat down, cover and let it simmer for 10-15min.
\end{itemize}

\paragraph{Directions (Sauce):}
\begin{itemize}
	\item Mix all the meatball ingredients and make small meatballs.
	\item Add a bit of olive oil to a pan, sear the meatballs in high heat both sides, then add the tomato sauce and spice, lower the heat down, cover and let it simmer for 10-15min.
\end{itemize}

\subsection{Lasagna}

\paragraph{Ingredients:}

\begin{itemize}
  \item Ground Beef Ingredients
  \item Bechamel Ingredients
  \item Lasagna sheets
  \item Mozzarella cheese
\end{itemize}

\paragraph{Directions:}
\begin{itemize}
  \item Make Ground Beef
  \item Make Bechamel
  \item Start layering: Sheets, Beef, Sheets, Bechamel, Mozzarella, Sheets, ... You should make 3 layers of beef. In the top, cover with mozzarella cheese.
\end{itemize}

Note: You can replace the ground beef for pesto (I’ve used the costco one) for a Pesto Lasagna.

\subsection{Beef Empanadas}
\paragraph{Ingredients:}
\begin{itemize}
  \item Ground Beef Ingredients
  \item 1/2 tsp oregano dry
  \item 1 tsp smoked paprika
  \item 1 tsp chili powder
  \item 1/2 tsp kosher salt
  \item 1 tsp brown sugar
  \item 2 tbl hot sauce (I’ve used Cholula)
  \item beef stock
  \item 1 egg for egg wash
\end{itemize}

\paragraph{Directions:}
\begin{itemize}
  \item Make ground beef, or using the ground beef left over
  \item Mix all spices (oregano, smoked paprika, chili powder, salt, brown sugar, hot sauce) using beef
stock as needed to allow stirring.
  \item Working with one puff pastry sheet at a time, roll it out so that it’s long enough to cut 6 circles, 2 on each
row. Using a 4 inch cookie cutter or bowl, cut 6 circles, 2 per row. Fill each with a about a tablespoon
of meat mixture in the middle of each pastry round. Brush half of the pastry round edge with egg wash,
then fold the dough over into a half-moon shape so the edges meet and then press them together with
your fingers to seal. Use a fork to crimp the edge. Place it onto a baking sheet and repeat with remaining
ingredients.
  \item Poke holes in each empanada using a fork then brush the empanadas with the egg wash and bake (on 375-400F) for about 20 to 25 minutes, or until golden brown.
\end{itemize}


\subsection{Sheperds Pie}
\paragraph{Ingredients (For Meat Mixture):}
\begin{itemize}
  \item 1 tbsp olive oil
  \item 1 1/4 lb lean ground beef
  \item salt and pepper to taste
  \item 1 large onion chopped
  \item 1 clove garlic minced
  \item 1/2 tsp red pepper flakes
  \item 2 tbsp Worcestershire sauce
  \item 1.9 oz onion soup mix I used Lipton, 55g pkg
  \item 1 cup beef broth
  \item 2 cups frozen veggies I used mix of peas, carrots, green beans and corn
\end{itemize}

\paragraph{Ingredients (Mashed Potatoes):}
\begin{itemize}
  \item 6 large potatoes
  \item 1 tbl spoon butter
  \item 2 tbl spoon powdered milk (I’ve used nido)
  \item 2 tbl spoon Parmesan cheese
\end{itemize}

\paragraph{Directions:}
\begin{itemize}
  \item Start by first cooking the potatoes in boiling water for about 15 minutes or until fork tender. While the
potatoes are cooking, you can prepare the meat mixture.
  \item Heat the oil in a large skillet over medium heat. Add the ground beef to the skillet and cook it for abut
5 minutes or until it’s no longer pink, breaking it up as you go along.
  \item Add the onion and garlic and cook for 3 more minutes until the onion softens and becomes translucent.
  \item Taste for salt, might need more.
  \item Add the pepper flakes, Worcestershire sauce, onion soup mix, beef broth and stir. Stir in the frozen
veggies and cook for a couple more minutes. Set aside.
  \item Preheat the oven 350 F degrees.
  \item Prepare the mashed potatoes.
  \item Spread the potatoes over the meat and smooth with a spoon. Take a fork and rough up the top a bit and
garnish with a bit of parsley.
  \item Place the skillet on a baking sheet, then place it in the oven and bake for 40 minutes until golden brown
on top.
  \item Garnish with more parsley and pepper and serve warm.
\end{itemize}


\section{Chicken}

\subsection{Breaded chicken}

\paragraph{Ingredients:}

\begin{itemize}
	\item Panko breadcrumbs
	\item Good Seasons Salad Dressing Recipe Mix, Italian
	\item Egg whites
	\item Chicken Breast
	\item Lime
	\item Parchment paper
\end{itemize}

\paragraph{Directions:}
\begin{itemize}
	\item Cut the chicken breast in half, use a “hammer” to flatten it out and put lemon juice.
	\item Let it aside for 5-10min
	\item Turn on the oven on 450F
	\item Wash chicken on water
	\item Mix the panko breadcrumbs with the italian season.
	\item Brush the chicken on egg whites, then on the panko breadcrumbs.
	\item Put the parchment paper on a tray and then on the oven.
	\item Leave it on the oven for 20-30min or until fully brown.
\end{itemize}

Notes: You can replace the panko breadcrumb with tempura batter. The ingredients are: Tempura mix, cold water, salt and water.


\subsection{Chicken Piccatta}

\paragraph{Ingredients (Sauce):}
\begin{itemize}
	\item 3/4 cup of chicken stock
	\item 2 tbl spoon Butter
	\item 1/2 cup of white wine
	\item 1/2 cup of lemon juice
	\item 1/4 cup of capers
\end{itemize}

\paragraph{Ingredients (Chicken):}
\begin{itemize}
	\item Chicken breast w/ skin
	\item 1 cup and 1 tbl spoon Flour
\end{itemize}

\paragraph{Directions (Chicken):}
\begin{itemize}
	\item Season chicken with salt in both sides
	\item Apply flour in both sides of the chicken
	\item Add 1 tbl spoon of gee or butter with some olive oil in a skillet (don't use a non-stick one).
	\item Once oil is hot, add chicken skin side down.
	\item Sear for 3-5' until side golden brown, then flip.
	\item Once the other side is done, set chicken aside.
\end{itemize}

\paragraph{Directions (Sauce):}
\begin{itemize}
	\item Add wine and scrape the bottom of the pan so that it removes chicken bits (Deglaze)
	\item Add chicken stock and lemon juice
	\item Add capers
	\item Make a small mixture of 1 table spoon of butter and 1 table spoon of flour
	\item Use a whisk to add tiny bits of the mixture until sauce thickens.
	\item Taste sauce and add salt and pepper as necessary.
\end{itemize}

\subsection{Skillet Creamy French Mustard Chicken}

\paragraph{Ingredients:}
\begin{itemize}
	\item 1 tablespoon olive oil
	\item 2 pounds bone-in chicken thighs
	\item Kosher salt
	\item Ground black pepper
	\item 1/2 cup diced shallots
	\item 1/2 cup dry white wine
	\item 1 cup low-sodium chicken broth
	\item 2 tablespoons whole-grain Dijon mustard
	\item 2 tablespoons Dijon mustard
	\item 2 tablespoons heavy cream	
\end{itemize}

\paragraph{Directions:}
\begin{itemize}
	\item Pat the chicken dry with paper towels and season with salt and pepper.
	\item Add the chicken skin-side down in a large cast iron with oil and cook until the fat is rendered and the
	skin is crisp and golden-brown, 6 to 8 minutes. Reserve chicken.
	\item Add the shallots to the pan and cook over medium heat until softened, about 3 minutes.
	\item Add the wine, scrape up any browned bits at the bottom of the pan with a wooden spoon, and cook
	until evaporated, about 3 minutes.
	\item Stir in the broth and whole-grain mustard and bring to a simmer.
	\item Return the chicken skin-side up, and add any juices accumulated on the plate to the pan.
	\item In a 400F oven, braise the chicken for 15-20’ or until internal temperature reaches 165F.
	\item Transfer the chicken with tongs to a platter.
	\item Place the pan over medium-high heat, whisk the smooth Dijon mustard into the sauce, and simmer until
	reduced slightly, about 2 minutes. Add the cream.
	\item Adjust salt and pepper accordingly.	
\end{itemize}

\subsection{Strogonoff}

\paragraph{Ingredients:}

\begin{itemize}
	\item 1 grated onion
	\item 800g of meat cut into cubes
	\item 4 tbl spoons of ketchup
	\item 1 tbl spoon of Worcestershire sauce
	\item 1 tbl spoon of mustard (the yellow cheap one)
	\item 2 tbl spoon of oil
	\item 1/2 cup of whipped cream
	\item 1 can of mushrooms
\end{itemize}

\paragraph{Directions:}
\begin{itemize}
	\item Season the meat with salt and pepper. Let it sit for 10-15'
	\item In the high heat, add onion, then throw the meat. 
	\item Once the meat is seared, lower the heat, add ketchup, mustard, Worcestershire sauce and mushrooms.
	\item Let it simmer for 10'
	\item Add whipped cream and simmer for 3-5'. Be careful not to boil.
\end{itemize}

\paragraph{Notes:}
\begin{itemize}
	\item You can make a chipotle Strogonoff by using 1 table spoon of chipotle pepper (Best brand is Embasa: Chipotle peppers in adobo sauce) instead of the ketchup, worcestershire sauce and mustard.
	\item You can replace the whipping cream with 1 1/2 cup of milk and 2 tsp of cornstarch. But you have to thicken the milk by mixing on the heat for a few minutes.
\end{itemize}


\subsection{Chicken in White Wine Sauce with Mushrooms}

\paragraph{Ingredients:}

\begin{itemize}
	\item 2 tbsp butter
	\item 1 lb chicken breasts boneless and skinless cut in half lengthwise
	\item salt and pepper to taste
	\item 1 medium onion chopped
	\item 3 cloves garlic minced
	\item 12 oz white mushrooms sliced, 340 g
	\item 1 tbsp all-purpose flour
	\item 1/4 cup white wine
	\item 1 cups whole milk
	\item parsley for garnish
\end{itemize}

\paragraph{Directions:}
\begin{itemize}
	\item Add the butter to a large skillet and melt over medium high heat.
	\item Season chicken breasts on both sides with salt and pepper.
	\item Place chicken breasts in skillet and cook on both sides, about 5 min per side or until no longer pink inside. Remove chicken from skillet.
	\item Add onion and garlic to skillet and cook for a couple minutes until onion is translucent and soft. Add mushrooms and stir. Season mushrooms generously with salt and pepper. Let cook for about 5 minutes, stirring occasionally. When mushrooms are cooked to your liking, reserve them along with the onions.
	\item De glaze pan with wine, let the alcohol cook.
	\item Whisk milk and flour and pour into skillet
	\item Add chicken and mushrooms into the skillet, mix everything and simmer for 3'.
	\item Garnish with parsley and serve hot.
\end{itemize}

\subsection{Chicken Shawarma}

\paragraph{Ingredients:}

\begin{itemize}
	\item 1/2 cup olive oil
	\item 1/4 cup lemon juice from 2 lemons
	\item 2 tsp smoked paprika
	\item 1/2 tsp tumeric
	\item 2 tsp cumin powder
	\item 1/2 tsp cinnamon
	\item 4 cloves garlic minced
	\item 1/2 tsp salt
	\item 1 large onion sliced
	\item 2 lbs chicken thighs boneless and skinless
	\item 2 tbsp fresh parsley for garnish
	\item chicken stock
\end{itemize}

\paragraph{Directions:}
\begin{itemize}
	\item Add the olive oil, lemon juice, paprika, turmeric, cumin, cinnamon, red pepper flakes, garlic, pepper, salt and onions to a ziploc bag, then add the chicken thighs. Seal the bag and shake it well to mix the ingredients. Place the bag in the refrigerator and marinate for at least 1 hour.
	\item Transfer the chicken into a pressure cooker and add chicken stock enough to cover the chicken.
	\item Cook under pressure for 10-15'
\end{itemize}

\subsection{Chicken Madeira}

\paragraph{Ingredients:}
\begin{itemize}
	\item 4 chicken breasts skinless and boneless, about 1 lb
	\item salt and pepper to taste
	\item 3 tbsp olive oil
	\item 1 lb cremini mushrooms cleaned and sliced
	\item 2 cloves garlic minced
	\item 2 cups Madeira wine or a dry red wine
	\item 1 cup chicken broth
	\item 1 tbsp all-purpose flour optional
	\item 2 tbsp butter
	\item 2 tbsp fresh parsley chopped
\end{itemize}

\paragraph{Directions:}
\begin{itemize}
	\item Season the chicken breasts generously with salt and pepper.
	\item Heat 2 tbsp of the olive oil in a large skillet or a saucepan over medium-high heat. Add the chicken
	breasts to the skillet and cook for about 3 to 4 minutes per side until they get to get golden brown. More
	time may be needed depending on the thickness of your chicken breasts. Remove the chicken breasts from
	the skillet and set aside.
	\item Add the remaining 1 tbsp of olive oil to the skillet and add the mushrooms. Season the mushrooms with
	salt and pepper then cook for about 8 minutes until the mushrooms start to brown. Stir occasionally.
	\item Add the garlic, Madeira wine and chicken broth to the skillet and stir. Season with more salt and pepper
	as needed. Reduce heat and cook for another 15 minutes until the sauce thickens a bit and reduces.
	\item If you find that the sauce hasn’t thickened enough, you can take about a ladle of the liquid from the pan
	and whisk it with a tbsp of flour, then pour it back into the saucepan and stir, the sauce should thicken
	almost instantly. Add the butter and stir, this will give the sauce a nice glossy color.
	\item Add the chicken breasts back to the pan and cook for another 5 minutes.
	\item Garnish with fresh parsley and serve over mashed potatoes.	
\end{itemize}

\paragraph{Notes:}To make the chicken madeira, you just need to change the wine to Masala wine (1 1/2 cup) + heavy cream (1 1/2 cup) and 3 cups of chicken stock.

\subsection{Butter Chicken}

\paragraph{Ingredients:}
\begin{itemize}
	\item 3 tbsp butter unsalted
	\item 8 cloves garlic minced
	\item 2 tsp fresh ginger minced (or paste)
	\item 1 cup tomato puree or passata
	\item 2 tbsp tomato paste
	\item 3 tsp garam masala
	\item 1 tbsp coriander ground
	\item 1 tsp cumin ground
	\item 1 tbsp smoked paprika
	\item 1 tsp turmeric
	\item 1 tsp salt
	\item 2 lbs chicken thighs boneless and skinless, cut into pieces
	\item 1 cup water
	\item 1 cup whole cream
	\item 1 tbsp flour
	\item 2 tbsp parsley chopped	
\end{itemize}

\paragraph{Directions:}
\begin{itemize}
	\item Add the butter and cook until the butter has melted. Add the garlic and ginger and saute for another
	minute or until the garlic becomes aromatic. Do not cook too long because you don’t want to burn it.
	\item Add the tomato puree and tomato paste to the Instant Pot and stir. Add the garam masala, coriander,
	cumin, paprika, turmeric and salt to the Instant Pot, stir and cook for about 3 to 5 minutes.
	\item Add the chicken thighs, water and stir everything together. There should be enough liquid in the pot to
	cover the chicken, so add more water if needed.
	\item Close the lid. Cook on the pressure cooker for 20’.
	\item Whisk milk and flour and add to the pot. Simmer for another 5’ until sauce reduces a bit. Add chopped
	parsley.	
\end{itemize}

\subsection{Italian Pasta Chicken - Mo Pasta}

\paragraph{Ingredients (For seasoning chicken):}
\begin{itemize}
	\item 1 lb Chicken chopped into small pieces
	\item 1-2 tsp Knoor
	\item 6-7 Garlic cloves, chopped
\end{itemize}

\paragraph{Ingredients (For sauce):}
\begin{itemize}
	\item 1 tbl spoon tomato paste
	\item 1 cup of water
	\item 2 onions chopped
	\item 1 bouquet garni (Thyme, parsley, bay leaf)
\end{itemize}

\paragraph{Directions:}
\begin{itemize}
	\item Sear the chicken, both sides. 5-10min.
	\item Add sauce in, season with salt and pepper.
	\item Let it simmer for 10-15min
\end{itemize}

\section{Sandwich}

\subsection{Croque Madame}

\paragraph{Ingredients (Bechamel):}
\begin{itemize}
	\item 1 tbl spoon flour
	\item 1 tbl spoon butter
	\item 1 pinch nutmeg
	\item 2/3 cup of milk
\end{itemize}

\paragraph{Ingredients (Sandwich):}
\begin{itemize}
	\item 4 loafs of bread
	\item 2 tbl spoon butter
	\item Ham
	\item 2 eggs
	\item 4 slices of gruyere cheese
\end{itemize}

\paragraph{Directions (Bechamel):}
\begin{itemize}
	\item Prepare bechamel sauce using Bechamel ingredients
\end{itemize}

\paragraph{Directions (Sandwich):}
\begin{itemize}
	\item Prepare bechamel sauce using Bechamel ingredients
	\item Toast lighly the bread first (spread butter on loaf)
	\item Once lighly toasted add cheese on and let it melt on each side of toast
	\item Cook egg sunny side up
	\item Order of ingredients are: Egg / (Sauce / Bread) / Cheese / Ham / (Cheese / Bread)
\end{itemize}


\subsection{Mustard Sandwich}

\paragraph{Ingredients:}

\begin{itemize}
	\item 1 tablespoon of mayo 
	\item 1 teaspoon of Dijon mustard 
	\item 2 slices of bread 
	\item 3 slices of ham 
	\item 3 slices of provolone
\end{itemize}

\paragraph{Directions:}
\begin{itemize}
	\item Spread mayo and mustard on the inside part of the bread 
	\item put 3 slices of ham inside the bread 
	\item put 3 slices of provolone in the top of the bread 
	\item put 10min @ 450F on the toaster 
\end{itemize}

\section{Pastas}

\subsection{Fettuccine Alfredo (Olive garden)}

\paragraph{Ingredients:}

\begin{itemize}
  \item Parmesan Grated (3/4 cup)
  \item Milk (1/2 cup - 8 tbl spoons)
  \item 1/2 cup butter
  \item Philadelphia cheese (8oz)
  \item Salt
  \item Pepper
\end{itemize}

\paragraph{Directions:}
\begin{itemize}
  \item Melt butter, then mix all ingredients.
  \item Add salt and pepper to taste
\end{itemize}

\subsection{Smoked Salmon Alfredo}

\paragraph{Ingredients:}

\begin{itemize}
	\item 1 cup of whipping cream
	\item 1/2 cup of parmesan cheese
	\item pinch of salt
	\item ground pepper to taste
	\item 1 package of trader joes salmon (4oz) chopped
	\item 1/2 chopped onion
	\item 1 tbl spoon of butter
	\item 1 tsp of oil
\end{itemize}

\paragraph{Directions:}
\begin{itemize}
	\item In a skillet add a bit of butter and oil
	\item Add onions
	\item Let it caramelize
	\item Add smoked salmon
	\item Add whipping cream and parmesan
	\item Taste for salt, add if necessary
	\item Add ground pepper to taste
\end{itemize}

\subsection{Pasta al Fungi}

\paragraph{Ingredients:}

\begin{itemize}
	\item 2 cups Dried Porcini Mushrooms
	\item 2 cups heavy cream
	\item 1 1/4 cup parmesan cheese
	\item 2 table spoon butter
	\item 3 table spoons flour.
\end{itemize}

\paragraph{Directions:}
\begin{itemize}
	\item Put dried porcini on hot water (1 cup) and let it rest for 1hr. Then boil the water until it reduces the water significantly.
	\item On a saucepan add ground pepper and butter, let it boil for a little bit.
	\item Add heavy cream, and parmesan.
	\item Add mushroom and half the liquid.
	\item On a small container add the other half of the mushroom liquid and flour, mix well.
	\item Mix flour paste and main ingredients.
	\item Taste and if necessary add salt and/or pepper.
\end{itemize}

\subsection{Fettuccine al Tartufo}

\paragraph{Ingredients:}

\begin{itemize}
	\item 2 tbl spoons of White Truffle Olive Oil
	\item 2 tea spoons of truffle oil salt
	\item 2 cups heavy cream
	\item 1 1/4 cup parmesan cheese
	\item Milk to taste
	\item 2 table spoon butter
	\item 3 table spoons flour.
\end{itemize}

\paragraph{Directions:}
\begin{itemize}
	\item On a saucepan add ground pepper and butter, let it boil for a little bit.
	\item Add heavy cream, and parmesan.
	\item Add Truffle oil and salt
	\item Mix flour paste and main ingredients.
	\item Taste and if necessary add milk if too salty or truffle salt if not enough salty.
\end{itemize}

\subsection{Carbonara}

\paragraph{Ingredients:}

\begin{itemize}
	\item 2 large eggs
	\item Olive oil
	\item Pancetta or bacon
	\item 4 garlic cloves chopped
	\item 1 cup of parmesan
	\item Ground pepper
	\item Parsley
	\item Spaguetti
\end{itemize}

\paragraph{Directions:}
\begin{itemize}
	\item Mix first the eggs, parsesan and ground pepper (You might add a touch of pecorino cheese as well). Add a pinch of salt.
	\item Heat the skillet and put in either the bacon or pancetta. (if using pancetta you need a bit of olive oil). After the bacon is ready add the garlic for 1min.
	\item When the pasta is ready reserve 1 cup of the water for later
	\item Mix the pasta, and mixture done in step 1 quickly so that the egg cooks (make sure to mix outside the oven so the eggs don’t become scrambled), then mix in the fat, bacon and garlic from step 2. Add the reserved water if necessary.
	\item Season with parmesan, salt and pepper for taste.
	\item Garnish with parsley
\end{itemize}


\subsection{Brother-in-law’s Cheese Sauce}

\paragraph{Ingredients:}

\begin{itemize}
	\item 1/3 to 1/2 of smoked provolone
	\item 1/2 cup of shredded parmesan
	\item 3 cups of heavy cream (consider mixing with milk to take richness)
	\item 2 onions chopped into small pieces
	\item 2/3 of the amount of onion in bacon
	\item 4 garlic chopped
\end{itemize}

\paragraph{Directions:}
\begin{itemize}
	\item Fry bacon, once ready reserve, keep bacon fat in pan.
	\item Fry onion, once caramelized, add cream and cheese
	\item Last add the bacon bits
\end{itemize}

\section{Desserts}

\newcommand{\mousse}[3]{
\subsection{Chocolate Mousse}

\paragraph{Ingredients:}

\begin{itemize}
	\item 1 can of condensed milk
	\item 1 can of "Crema de leche"
	\item #2
\end{itemize}

\paragraph{Directions:}
\begin{itemize}
	\item Blend all ingredients on a blender.
\end{itemize}

\paragraph{Note: #3}
}

\mousse{Cheese}{Philadelphia cheese or brazilian requeijao, Guava Jam}{Melt the guava jam and mix after mixing all the ingredients}

\mousse{Passion Fruit}{Passion Fruit Juice}{Melt the guava jam and mix on the mixture after mixing all the ingredients}

\subsection{Banana Bread}

\paragraph{Ingredients:}

\begin{itemize}
	\item 1 stick of butter
	\item 3/4 cup of sugar
	\item 1 egg
	\item 3 ripe bananas
	\item 1 pinch of salt
	\item 1 1/2 vanilla extract
	\item 2 cups of flour
	\item 1 teaspoon of baking soda
	\item 1/2 cup of chocolate chips
\end{itemize}

\paragraph{Directions:}
\begin{itemize}
	\item Beat butter, sugar and eggs
	\item Smash bananas and add them
	\item Add remainder ingredients
	\item Bake at 350F for 45min
\end{itemize}


\subsection{Banoffee Pie}

\paragraph{Ingredients (Massa):}
\begin{itemize}
	\item 5 bananas nanicas bem madura
	\item 2 latas de leite condensado cozido ou doce de leite pronta (a da kitanda da bem)
	\item 1 pacote de 400 de bolacha de sua prefeência
	\item 200g de manteiga ou margaria
	\item 1 colher de acucar
	\item chantily para decorar (heavy cream + acucar)
	\item canela em pó para decorar
\end{itemize}

\paragraph{Directions (Massa):}
\begin{itemize}
	\item Bate a bolacha, mistura com a manteiga (Depois de derreter no microondas) e adiciona a acucar
	\item Coloca na forma
	\item Bota na geladeira para descansar uns 30min
\end{itemize}

\paragraph{Ingredients (Montar):}
\begin{itemize}
	\item
\end{itemize}

\paragraph{Directions (Montar):}
\begin{itemize}
	\item Bota o doce de leite por cima da massa.
	\item Depois cortar as bananas em rodela, colocar em cima do doce de leite
	\item Coloca o chantily
	\item Polvilha com canela
\end{itemize}




https://www.youtube.com/watch?v=OWh748JNruM

\subsection{Creme brulee}

\paragraph{Ingredients:}

\begin{itemize}
\item 1 tsp vanilla bean paste
\item 6 egg yolks
\item 6 tbl spoon of sugar
\item 2 1/2 cup of heavy cream
\end{itemize}

\paragraph{Directions:}
\begin{itemize}
\item Combine heavy cream and sugar in a sauce pan. Place over medium heat, bring just to a simmer and remove from heat.

\item Whisk the egg yolks until they lighten in color in a mixing bowl. Slowly add the cream mixture, mixing continuously.
\item Add back to the sauce pan on medium heat, stir constantly until mixture thickens. About 5'. Don't let it boil.
\item Put mixture in small containers and bake in a preheated oven (300F) for 30'. 
\item Let it refrigerate overnight
\item Using a hand torch add a bit of sugar over the top and caramelize the sugar. Be careful not to burn.
\end{itemize}

Notes: Try with 5 eggs and more vanilla paste (2-3 tsp?)

\subsection{Milk Pudding (Pudim de leite)}

\paragraph{Ingredients:}

\begin{itemize}
	\item 1 can of condensed milk
	\item 1 1/2 cup of sugar
	\item Milk
	\item 1 1/2 spoon of cornstarch
\end{itemize}

\paragraph{Directions:}
\begin{itemize}
	\item Blend the condensed milk, same amount of milk and cornstarch.
	\item On a saucepan, melt sugar. It takes quite a while
	\item In a pudding container, carefully put the melted sugar. It gets solid pretty quickly.
	\item Turn on heat at 350F, setup a bain-marie.
	\item Wrap pudding in aluminium foil, let it cook for 30-40'
	\item Remove aluminium foil and cook until ready (When its not wet inside)
\end{itemize}

\subsection{Beet Pistachio Bars}

\paragraph{Ingredients (Dry components):}
\begin{itemize}
	\item 1/4 cup brown sugar
	\item 1 cup brown rice flour
	\item 1/4 cup coconut flour
	\item 2 teaspoons baking powder
	\item 2 teaspoon cinnamon
	\item 1/2 teaspoon salt
	\item 1/2 cup dark chocolate chips (optional)
	\item 1/2 cup unsalted shelled pistachios
	\item Zest of 1 lemon
\end{itemize}

\paragraph{Ingredients (Wet components):}
\begin{itemize}
	\item 2 large eggs
	\item 1/3 cup melted coconut oil
	\item 1/2 cup low-fat milk
	\item Some oil for roasting the beets, e.g. sunflower/avocado oil.
	\item 1 pound beets (about 4 medium-sized), peeled and chopped
\end{itemize}

\paragraph{Directions:}
\begin{itemize}
	\item Add a little bit of oil to chopped beets, mix well.
	\item Roast beets on 400F (200C) for 35'. Let them cool.
	\item On a blender add all wet ingredients (Except eggs)
	\item Once blended, move to a container and add eggs.
	\item On a separate container, mix the dry ingredients.
	\item Mix both containers well.
	\item Using a baking pan, add parchment paper and spread evenly. Bake for 30'.
\end{itemize}

\subsection{Brownie Sofia}

\paragraph{Ingredients:}

\begin{itemize}
	\item 2 packages of 10oz bittersweet chocolate (Ghirardelli premium baking chips - 60%)
	\item 4 tablespoons unsalted butter
	\item 1 cup of fine granulate sugar
	\item 2 eggs
	\item 1/2 tsp of vanilla extract
	\item 2/3 cup flour
	\item 1 cup of nuts
	\item a touch of salt
\end{itemize}

\paragraph{Directions:}
\begin{itemize}
	\item Melt the chocolate and the butter altogether (You can melt on the microwave, perhaps 30-60'' at a time)
	\item Let it cool a little bit.
	\item Add everything (except the flour), mix well.
	\item Add flour.
	\item In a baking sheet, spread butter and flour.
	\item Spread mixture well in baking sheet.
	\item Put in the oven for 30min at 375F.
\end{itemize}

\section{Cured Meats}

\subsection{Charque}

Originals

https://www.youtube.com/watch?v=zCUlzI2n0ic
https://www.youtube.com/watch?v=QEAqYDH6DDk
https://www.youtube.com/watch?v=liP9mfqCIQw

\paragraph{Ingredients:}

\begin{itemize}
	\item 2lb Lagarto meat -- Trim the excess fat
	\item liquid smoke
	\item Non iodized salt
\end{itemize}

\paragraph{Directions:}
\begin{itemize}
	\item Season the meat with liquid smoke, don't over do it
	\item Cover with salt and massage thorough
	\item Add to the fridge
\end{itemize}

\paragraph{Directions (T+1 day):}
\begin{itemize}
	\item Remove salt and add more salt
	\item Repeat the process until there is not a lot of water being released
\end{itemize}

\subsection{Coppa}

\paragraph{Ingredients:}

\begin{itemize}
	\item 1000 g pork neck coppa	
	\item 27.5 g kosher salt 2.75\%
	\item 2.5 g Cure \#2 0.25\%
	\item 0.45 g black pepper cracked; 0.045\%
	\item 0.25 g cloves ground; 0.025\%
	\item 0.1 g bay leaf ground; 0.01\%
	\item 0.15 g cinnamon ground; 0.015\%
	\item 0.1 g nutmeg ground; 0.01\%
\end{itemize}

\paragraph{Directions:}
\begin{itemize}
	\item Trim the meat into a boneless, uniform shape. Make sure there are no cuts in the meat where bacteria could enter, and cut off any loose pieces.
	\item Weigh the coppa in grams. Divide by 1000, then multiply each ingredient by that number. For example, if your coppa weighs 2650 g, you need to multiply the ingredients specified above by 2.65.
	\item Mix all the salt and the seasonings together, and rub on the meat. Place the coppa and all the extra salt and seasonings in a vacuum-sealable bag and seal. You can also use a Ziploc bag.
	\item Place the bag in the fridge for 7 days. Flip the bag every day or so.
	
	\item After 7 days have passed, remove the meat from the bag and gently scrape off any excess salt and seasonings.
	\item Prepare the coppa for hanging by casing (veil, beef bung) and trussing. Poke a lot of small holes all over the surface with a sterilized needle to remove any trapped air.
	\item Spray with white mold solution.
	\item Weigh the meat and write it down on a tag. Attach the tag to the meat.
	\item Hang and dry in the curing chamber at 55F and 75\% humidity .
	\item Make sure to check often the dehumidifier container so that it doesn't fill up.
	\item Once the meat loses 35\% of the weight take off curing chamber.
	\item Vacuum seal and put on the fridge for 7 days.
	\item Once the time passed, use a slicer for cutting the coppa and enjoy.
\end{itemize}

Original: https://tasteofartisan.com/capicola-recipe/

\section{Sauces}

\subsection{Tomato Sauce (Sergio)}

\paragraph{Ingredients:}

\begin{itemize}
	\item 1 onion 
	\item 1 tomato 
	\item 3 cloves of garlic 
	\item 1 tomato paste can
\end{itemize}

\paragraph{Directions:}
\begin{itemize}
	\item Chop onion, tomatoes and garlic. On a sauce pan add one table spoon of oil, mix veggies and let it sauté for a while (until smells). 
	\item Remove from sauce pan, put on blender and mix all ingredients. 
	\item Put the blended mix back into the sauce pan. Reduce the fire, mix in tomato paste and let it simmer for 30'.
\end{itemize}

\subsection{Teriyaki Sauce}

\paragraph{Ingredients:}

\begin{itemize}
	\item 1 cup of water
	\item 4 table spoon of brown sugar
	\item 1/4 cup soy sauce
	\item 1 tablespoon honey
	\item 1 large clove of garlic, finely minced
	\item 1/2 teaspoon ground ginger
	\item 2 tablespoons cornstarch
\end{itemize}

\paragraph{Directions:}
\begin{itemize}
	\item Mix everything, put on low heat, mix and wait until its thick.
\end{itemize}

\subsection{Bechamel Sauce (Blue Apron)}

\paragraph{Ingredients:}

\begin{itemize}
\item 1 tablespoon of butter
\item 1 tablespoons all purpose flour
\item 1 cup of whole milk at room temperature
\item 1/2 cup fontina cheese
\item Optional: Nutmeg
\end{itemize}

\paragraph{Directions:}
\begin{itemize}
\item Melt the butter over medium heat
\item Add flour and whisk until smooth (2min)
\item Add the milk gradually
\item Simmer until thick enough (~10min)
\item Optional: Add nutmeg
\item Add the fontina cheese
\item Season with salt and pepper
\end{itemize}

\subsection{Bechamel Sauce (Adapted Luzinete Style)}

\paragraph{Ingredients:}

\begin{itemize}
	\item 2 tablespoons of flour
	\item 1 onion or garlic (2 cloves)
	\item 1 cup of milk
	\item (Optional) pinch of nutmeg
	\item 1 tablespoon of butter
	\item 1 1/2 tablespoon of Parmesan cheese
\end{itemize}

\paragraph{Directions:}
\begin{itemize}
	\item On a blender add the milk and flour
	\item Put the butter on a sautee pan, add grated onion or garlic
	\item Add mixture of milk and flour
	\item Wait until thickens
	\item Add Parmesan cheese and nutmeg
\end{itemize}

\subsection{Roux}

\paragraph{Ingredients:}

\begin{itemize}
	\item 2 tbl spoon clarified butter (Melted butter where the fat is separated from the milk)
	\item 2 tbl spoon of white flour
\end{itemize}

\paragraph{Directions:}
\begin{itemize}
	\item Melt clarified butter (if not already melted)
	\item Add flour and whisk constantly
	\item For white roux, around 5' will cause flour to lose that raw smell. We are looking for a wet sand consistency. About 20' for blonde roux (smell of toasted bread) and 35' for brown roux (Peanut butter). Dark roux ~ 45'.
\end{itemize}

\subsection{Mother sauces}

\paragraph{Ingredients:}
\begin{itemize}
	\item Bechamel: White roux mixed with milk
	\item Veloute (From velvet): White roux mixed with clear stock (usually chicken or vegetable).
	\item Espagnole: Dark roux + beef stock.
\end{itemize}

\paragraph{Directions:}
\begin{itemize}
	\item Notes: Mix ingredients (1:8 part ratio). For example: 2 table spoons of roux and 1 cup of liquid (1 cup = 16 tbl spoons), let it simmer for 10' and strain at the end.
\end{itemize}


\subsection{Pesto Sauce}

\paragraph{Ingredients:}

\begin{itemize}
	\item 2 garlic cloves
	\item 2 cups of basil
	\item 1/4 cup of pine nuts
	\item 1/3 cup of olive oil
	\item 1/2 Parmesan cheese
	\item salt to taste
\end{itemize}

\paragraph{Directions:}
\begin{itemize}
	\item Mix everything in a mixer. Add salt and pepper to taste.
\end{itemize}

Notes: You can make a sauce for pasta by adding a bit of milk and cornstarch (mix both together before putting in pesto mixture).

\subsection{Chipotle Mayo}

\paragraph{Ingredients:}

\begin{itemize}
	\item 2 eggs raw
	\item oil (Canola or Peanut oil)
	\item Chipotle peppers in adobo sauce (in the can)
	\item salt
	\item lime
\end{itemize}

\paragraph{Directions:}
\begin{itemize}
	\item Using a mixer, add the two eggs and oil until it emulsifies. Don't be shy on the oil.
	\item Use a spoon to see if the consistency reminds you of mayonnaise.
	\item Add chipotle peppers (You might want to half first otherwise it might be too spicy)
	\item Add the lime of 2 lemon wedges
	\item Add salt to taste
\end{itemize}

\subsection{Chimichurri}

\paragraph{Ingredients:}
\begin{itemize}
  \item 1/2 cup olive oil
  \item 1/4 cup red wine vinegar
  \item 1 tbsp lemon juice freshly squeezed
  \item 1/2 cup curly leaf parsley minced
  \item 1 tbsp dried oregano (or fresh)
  \item 1/2 tsp red pepper flakes
  \item 1/2 tsp salt
  \item 1/2 tsp black pepper freshly ground	
\end{itemize}

\paragraph{Directions:}
\begin{itemize}
  \item Mix all ingredients and let it sit in the fridge overnight.	
\end{itemize}

\section{Dump}

\subsection{Miso Dressing}

\paragraph{Ingredients:}

\begin{itemize}
	\item 1/3 cup of lemon juice
	\item 1/2 cup of olive oil
	\item 1 tbl spoon of miso paste
	\item pinch of salt
	\item 2 garlic cloves finely minced
\end{itemize}

\paragraph{Directions:}
\begin{itemize}
	\item Mix all ingredients
\end{itemize}

\subsection{Cesar Dressing (Blue Apron)}

\paragraph{Ingredients:}

\begin{itemize}
	\item 1 teaspoon of lemon zest
	\item 2 lemon wedges (the juice)
	\item 2 garlic cloves
	\item 1 tbl spoon white vinegar
	\item 1/4 cup mayo
\end{itemize}

\paragraph{Directions:}
\begin{itemize}
	\item Smash garlic until it resembles a paste, add vinegar and place it in a bowl. Let it marinate for 5-10'. 
	\item Mix all ingredients.
	\item Season with salt and pepper.
\end{itemize}

\subsection{Potato Salad}

\paragraph{Ingredients:}

\begin{itemize}
	\item 5 large potatoes 
	\item 2 ribs of celery, finely chopped 
	\item 1/2 onion, finely chopped 
	\item 3 hard boiled eggs. 2 chopped, 1 sliced 
	\item 1 cup miracle whip 
	\item 3 tablespoons Dijon mustard 
	\item salt 
	\item pepper
\end{itemize}

\paragraph{Ingredients (Miracle Whip):}

\begin{itemize}
	\item 6 teaspoons of white vinegar 
	\item 2 teaspoons of cornstarch 
	\item 3 teaspoon of sugar 
	\item 1 teaspoon paprika 
	\item 1 teaspoon of garlic salt 
	\item 1/4 teaspoon of mustard powder 
	\item 1 1/2 cup of mayo
\end{itemize}

\paragraph{Directions:}
\begin{itemize}
	\item Mix everything and season with salt and pepper to taste.
\end{itemize}

\subsection{Overnight Oats}

\paragraph{Ingredients:}

\begin{itemize}
	\item 1 cup of old fashioned oats
	\item 1 cup of almond milk
	\item 1 scoop of whey protein
	\item 1/2 tsp of pumping pie spice (1/2 tsp cinnamon, 1/8 tsp nutmet, 1/8 tsp ginger). I actually like more like 8:1:2 the proportions between (cinnamon, nutmeg and ginger).
	\item (Optional) 1 1/2 tsp maple syrup
\end{itemize}

\paragraph{Directions:}
\begin{itemize}
	\item Combine all ingredients, shake well. Leave in the fridge overnight.
\end{itemize}

\subsection{Roasted brussel sprouts}

\paragraph{Ingredients:}

\begin{itemize}
	\item Brussels sprouts
	\item Salt
	\item Pepper
	\item Olive oil
\end{itemize}

\paragraph{Directions:}
\begin{itemize}
	\item Cut brussels sprouts in half, mix with a bit of olive oil, salt and pepper.
	\item Turn oven on 475F and leave for 20'. Flip sides 10'.
\end{itemize}

Notes: You can make broccoli as well. Just add some chopped garlic to the broccoli.

\subsection{Spaguetti Squash}

\paragraph{Ingredients:}

\begin{itemize}
	\item Spaguetti Squash
	\item Olive oil
	\item Salt and pepper
	
\end{itemize}

\paragraph{Directions:}
\begin{itemize}
	\item Poke some holes on the squash and put on the microwave for 5'
	\item Remove from the microwave, cut it in half and remove the seeds with a spoon.
	\item Turn oven on 400F and leave it for 35-40'
	\item Using a fork, make the pasta
	\item Add your favourite sauce (marinara, olive oil and garlic, etc.)
\end{itemize}

\subsection{Tomato Soup}

\paragraph{Ingredients:}
\begin{itemize}
	\item 2 lb of tomatoes
	\item 7 garlic cloves
	\item 1 onion
	\item Chicken stock
	\item olive oil
	\item 3 table spoons Parmesan cheese
\end{itemize}

\paragraph{Directions:}
\begin{itemize}
	\item Cut tomatoes in half.
	\item Roughly chop onions
	\item Put all vegetables in a large tray.
	\item Season generously with salt and pepper
	\item Sprinkle olive oil on top
	\item Put tray on oven on 375F for 60-90minutes
	\item When vegetables are roasted put in a food mixer
	\item Put mixture in a sauce pan, add chicken stock until reaches desired consistency.
	\item Taste and depending on acidity add some sugar (1-2 tsp of coconut sugar). Make sure to balance with salt for sweetness.
\end{itemize}

\paragraph{Notes:}
\begin{itemize}
	\item (Optional) Add 1 tsp of cornstarch and 1 cup of milk for a richer soup.
	\item (Optional) Add 2-4 tbl spoon of Parmesan cheese.
\end{itemize}

\subsection{Beet Humus}

\paragraph{Ingredients:}

\begin{itemize}
	\item 1 can of whole beets
	\item 1 can of garbanzo beans
	\item 1 lemon (zest + juice)
	\item 6 garlic cloves
	\item 1/4 cup olive oil
\end{itemize}

\paragraph{Directions:}

\begin{itemize}
	\item Put everything in a mixer (Add some salt and pepper to taste) and mix. You might want to put some water too.
\end{itemize}

\subsection{Lemon and Sprite}

\paragraph{Ingredients:}

\begin{itemize}
	\item 2 lemons (the green)
	\item 1 sprite
	\item ice
\end{itemize}

\paragraph{Directions:}
\begin{itemize}
	\item Mix everything
\end{itemize}

\subsection{Caramelized Carrots}

\paragraph{Ingredients:}

\begin{itemize}
	\item 2 chopped carrots
	\item 2 cloves of garlic
	\item 1 tablespoon of honey or agave nectar
	\item 1 orange (Juice the orange)
\end{itemize}

\paragraph{Directions:}
\begin{itemize}
	\item  Heat 2 tablespoons of olive oil on a skillet
	\item Add carrots and garlic cook for 3-5 min until fragrant (you might want to cook carrots on steam before)
	\item Add agave nectar / orange and cook until dries out
	\item Season with salt and pepper to taste
\end{itemize}

\subsection{Requeijao Cremoso}

\paragraph{Ingredients:}

\begin{itemize}
	\item 2 litros de leite quente (fervendo)
	\item 10 colheres de vinagre branco
	\item 200ml de leite morno
	\item 8 colheres de manteiga
	\item 1 colher de sal
\end{itemize}

\paragraph{Directions:}
\begin{itemize}
	\item Ferva o leite, adicione o vinagre e misture por uns 30s.
	\item Deixe descansar por 5min.
	\item Separe o soro do leite com uma peneira e toalha
	\item Coloque o queijo no liquidificador com o leite morno a manteiga e o sal.
	\item Bata ate ficar com a consistencia cremosa.
\end{itemize}

\subsection{Ragu Clement}

\paragraph{Ingredients (Meat):}
\begin{itemize}
	\item 1.5 lb of meat
	\item 1/2 onion
	\item 3 garlic cloves
	\item 1 beef onion soup packet (Lipton)
	\item sugar
	\item chicken stock
\end{itemize}

\paragraph{Ingredients (White sauce):}
\begin{itemize}
	\item 1 Cup of milk
	\item 1 tbl spoon butter
	\item 1 tbl spoon flour
	\item Mozzarella
\end{itemize}

\paragraph{Directions (Meat):}
\begin{itemize}
	\item Add some oil into the pan, wait until it gets hot then add onion and garlic. Leave for 3-4min until onion trans-lucid and fragrant. Reserve.
	\item Rub sugar on sides of meat (Make small cuts of 2in x 5in) and sear both sides.
	\item After meat is seared, transfer to pressure cooker with reserved onion/garlic, add onion soup packet and cover with chicken stock until approximately 1in above the meat.
	\item Cook on pressure for 50'.
\end{itemize}

\paragraph{Directions (White sauce):}
\begin{itemize}
	\item Make a basic bechamel sauce with the milk, butter and flour.
	\item After the meat is done, shred the beef, put in a pirex and cover with the white sauce
	\item Put in the oven on 475 for 10' or until cheese melts.
\end{itemize}

\subsection{Blue Apron Sweet potatoes}

\paragraph{Ingredients:}

\begin{itemize}
	\item 1 lb sweet potatoes
	\item 1 tbl spoon white vinegar
	\item 1 shallot minced
	\item 1 bunch of chives thiny sliced
	\item Salt
	\item Pepper
	\item Olive oil
\end{itemize}

\paragraph{Directions:}
\begin{itemize}
	\item Place \textbf{chives, shallot and vinegar} in a bowl for marinate.
	\item Cook \textbf{sweet potatoes} in boiling water for 10-12min.
	\item After potatoes are cooked, drain them.
	\item Add olive oil to the same pan, add \textbf{marinate} and sear until onion is translucent.
	\item Add cooked \textbf{potatoes, mixture and season with salt and pepper}.
\end{itemize}

\chick{Orange and Soy}{2 tbl spoon olive oil, 1/2 cup of orange juice, 1/4 cup of soy sauce and 2 garlic cloves (pureed)}

\chick{KitchenSync Sauce}{2 tsp sugar, 1tsp salt, 1 tbl spoon garlic (or 2 minced garlic), 1 tbl spoon balsamic vinegar, 4 tbl spoon olive oil}

\chick{Paprika Chicken}{Knoor, Sweet Paprika (A lot - 2 tablespoons for 1 chicken breast)}

\chick{Fish sauce}{2 tbl spoon sugar, 2 garlic cloves (finely minced), fish sauce (1/4 cup)}


\subsection{Cauliflower Couscous}

\paragraph{Ingredients:}

\begin{itemize}
	\item 1/2 cauliflower
	\item 3-4 cloves of garlic
	\item Knoor
	\item Seasoning (Curry for example)
\end{itemize}

\paragraph{Directions:}
\begin{itemize}
	\item Grate cauliflower on a food processor for example.
	\item Wrap cauliflower in a towel to remove excess of water
	\item In a sauteed pan, put garlic for 2-3 min (until fragrant)
	\item Add salt, pepper and seasoning (curry for example) and let it cook for 6-7 min (Do not cover with lid otherwise will be watery)
\end{itemize}

\subsection{Creamy onion Chicken}

\paragraph{Ingredients:}

\begin{itemize}
	\item Garlic 
	\item Green onion 
	\item Chicken 
	\item Powdered Onion Soup (Lipton brand) 
	\item Parmesan cheese 
	\item Mozzarella cheese 
	\item Crème de leite (TODO)
\end{itemize}

\paragraph{Directions:}
\begin{itemize}
	\item Directions Stir fry the chicken with garlic and green onion. 
	\item Than put 1/4 of the package of the onion soup. Stir well.
	\item Then put the chicken with some mozzarella and parmesan cheese on the top. 
	\item Put in the oven on 400F for 10-15min.
\end{itemize}

\subsection{Bread (Rosinha)}

\paragraph{Ingredients:}

\begin{itemize}
	\item 1 cup of sweet potato - cooked and mashed
	\item 1/4 cup of oil
	\item 2 eggs
	\item 1 3/4 cup of gluten free flour (Bob Mills)
	\item 1 cup of brown sugar 
	\item 1/4 tsp of salt
	\item 1 tsp of baking soda (Try baking powder)
	\item 1 tsp of xanthan gum (Bob Mills)
	\item 1/3 cup of water (*try more water)
	\item 1 tsp of cinnamon (*try more)
\end{itemize}

\paragraph{Directions:}
\begin{itemize}
	\item Mix all ingredients, adding the water gently.
	\item Spread a bit of oil on a loaf pan (12 x 4 1/2 inches used in this recipe)
	\item Turn on the oven at 350F for 50'
	\item Check if bread is ready
\end{itemize}

Should yield 20 slices with 100 cal each.

\subsection{Filet Mignon - Luzinete Style}

\paragraph{Ingredients:}

\begin{itemize}
	\item Salt
	\item Pepper
	\item Filet Mignon
	\item Corn starch
	\item Red wine
	\item 2 onions (or 1 large)
\end{itemize}

\paragraph{Directions:}

For leaving overnight:
\begin{itemize}
	\item Wash the meat in water, grate the onion and put 1 cup of red wine. Add salt and pepper to taste.
	\item Put in a container (tupperware) and leave it overnight.
\end{itemize}


For the day the filet is being cooked:
\begin{itemize}
	\item In a large and hot skillet melt a tablespoon of butter
	\item Sear on all sides of the filet
	\item Remove the filet from the skillet, and in the same skillet add the misture (wine and onion) with 2 table spoons of cornstarch (Be careful with the cornstarch so that is fully dissolved)
	\item Let it cook for 5min
	\item Add the filet into the skillet and close it with a lid, put the fire on low and let it cook for 10min.
	\item Turn off the heat, cut the filet in pieces and use the juice from the meat and put back in the skillet. 
	\item Put 1 tablespoon of butter in the skillet and let it cook for a couple more minutes.
\end{itemize}

\subsection{Healthy Chicken Masala}

\paragraph{Ingredients:}

\begin{itemize}
	\item 1 onion chopped
	\item 1 teaspoon ginger
	\item 1 teaspoon coriander
	\item 1 teaspoon garam masala
	\item 1 teaspoon turmeric
	\item 1 teaspoon paprika
	\item 1 teaspoon salt
	\item 1 tablespoon tomatoe paste
	\item 2 chopped tomatoes
	\item 1/2 cup cilantro
	\item 1 tablespoon chopped garlic
	\item 2 chicken breasts
\end{itemize}

\paragraph{Directions:}
\begin{itemize}
	\item Mix all ingredients in the chicken, let it season for a day. 
	\item In a frying pan turn up the heat with oil and sear until ready.
\end{itemize}



\subsection{Chicken Fried Rice}

\paragraph{Ingredients:}

\begin{itemize}
	\item Cooked rice (Leave overnight on the fridge)
	\item 3 tablespoons peanut oil
	\item 2 tablespoons of soy sauce
	\item 1 teaspoon toasted sesame oil
	\item 2 eggs
	\item Chicken
	\item Chickpeas
\end{itemize}

\paragraph{Directions:}
\begin{itemize}
	\item Put peanut oil and crack the eggs
	\item Make an egg scramble then throw in the remaining ingredients
	\item Mix a bit and season with salt if required
	
\end{itemize}

Note: Experiment with more soy sauce + sesame oil + grated onions

\subsection{French Onion Soup}

\paragraph{Ingredients:}

\begin{itemize}
	\item 2 tbl spoons unsalted butter
	\item 3 large onions (2 pounds), halved lengthwise and thinly sliced crosswise
	\item sea salt
	\item fresh ground pepper
	\item 2 tablespoons dry sherry
	\item 1 quart rich beef stock
	\item 1 bouquet garni, made with 1 bay leaf, 1 thyme sprig, 2 juniper berries and 2 flat-leaf parsley sprigs, tied in cheesecloth
	\item 2 cups shredded Gruyère cheese (about 6 ounces)
\end{itemize}

\paragraph{Directions:}
\begin{itemize}
	\item Melt the butter in a large enameled cast-iron casserole. Add the onions and a pinch of salt, cover and cook over moderate heat, stirring once or twice, until the onions soften, about 10 minutes. Uncover and cook over moderate heat, stirring frequently, until the onions are lightly browned, about 40 minutes.
	\item Stir in the sherry. Add the stock and bouquet garni and bring to a boil. Cover and simmer over low heat until the soup has a deep flavor, about 30 minutes. Discard the bouquet garni and season the soup with salt and pepper.
	\item Preheat the oven to 425°. Bring the soup to a simmer, ladle it into 4 deep ovenproof bowls and sprinkle with half of the cheese. Bake the bowls of soup on a baking sheet in the middle of the oven for 10 minutes, or until the cheese is bubbling. Serve hot.
\end{itemize}

\subsection{Carrot Soup (Healthy)}

\paragraph{Ingredients:}
\begin{itemize}
	\item 1.5 lb of carrots (Chopped)
	\item 1 Onion
	\item Olive oil
	\item 2 cups of vegetable stock (no salt)
	\item 2 cups of water
	\item 1 table spoon of minced ginger
	\item 3 strips of orange
\end{itemize}

\paragraph{Directions:}
\begin{itemize}
	\item Put oven on 375 and add carrots and onion into a tray.
	\item Sprinkle olive oil and salt.
	\item Roast for 30-45' until slightly brown.
	\item Put onions, carrots, stock, water, ginger and orange on a saucepan and simmer for 30'
	\item Remove orange
	\item Blend everything
\end{itemize}

\subsection{Tomato Soup (Healthy)}

\paragraph{Ingredients:}
\begin{itemize}
	\item 3 lb of tomatoes
	\item 1 Onion
	\item Olive oil
	\item 2 cups of vegetable stock (no salt)
	\item 2 cups of water
\end{itemize}

\paragraph{Directions:}
\begin{itemize}
	\item Put oven on 375 and add tomatoes and onion into a tray.
	\item Sprinkle olive oil and salt.
	\item Roast for 30-45' until slightly brown.
	\item Put onions, tomatoes, stock and water simmer for 90'
	\item Blend everything
\end{itemize}

\subsection{Asparagus Soup (Healthy)}

\paragraph{Ingredients:}
\begin{itemize}
	\item 2 lb of asparagus
	\item 1 Onion
	\item Olive oil
	\item 2 cups of vegetable stock (no salt)
	\item 2 cups of water
	\item 1 table spoon of yogurt
\end{itemize}

\paragraph{Directions:}
\begin{itemize}
	\item Put oven on 375F and add onion into a tray.
	\item Sprinkle olive oil and salt.
	\item Roast for 30-45' until slightly brown.
	\item Put onions, asparagus stock and water simmer for 30'
	\item Blend everything
	\item Add yogurt
\end{itemize}

\subsection{BBQ Beans}

\paragraph{Ingredients:}

\begin{itemize}
	\item 1 cup de tomato sauce
	\item 1/2 cup de bbq sauce
	\item 2 table spoons of liquid smoke
	\item seasoning (e.g. 1 tbl spoon of knoor)
	\item 1lb of pinto beans
\end{itemize}

\paragraph{Directions:}
\begin{itemize}
	\item Add all ingredients into a pressure cooker and cook for 20'
\end{itemize}

\subsection{Rosbife Tia Tina (Tia Ana)}

\paragraph{Ingredients:}

\begin{itemize}
	\item manteiga
	\item limao
	\item pimenta branca
	\item sal
	\item alho
	\item molho ingles
\end{itemize}

\paragraph{Directions:}
\begin{itemize}
	\item Faz uns furinhos na carne pra deixar o tempero entrar. depois de algumas horas temperando eh so por na panela manteiga deixar bem quente e depois jogar o file.
	\item Sela o file numa panela aderente. Fica esfregando o file na panela. So vira o file quando tiver queimado de um lado. O segredo eh ficar esfregando o file na frigideira.
	\item Pra fazer o molho joga um pouco de molho ingles e agua e mistura as raspas que ficam na panela. prova e ve se precisa de sal. Fica provando pra ver se precisa de mais molho ingles.
\end{itemize}

\subsection{Stir Fried Chayote Squash (Sergio)}

\paragraph{Ingredients:}

\begin{itemize}
	\item 1 onion
	\item 3 garlic cloves
	\item 1 tomato
	\item Chayote Squash
	\item knoor 
\end{itemize}

\paragraph{Directions:}
\begin{itemize}
	\item Stir fry the onion and garlic with some oil until garlic is fragrant
	\item Add tomato and knor
	\item Add Chayote Squash (after being steamed)
\end{itemize}

\end{document}