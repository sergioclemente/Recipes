\documentclass{article}

\usepackage{listings}

\usepackage[a4paper, margin=0.8in]{geometry}

\title{Recipes}
\author{Sergio Clemente Filho}
\date{\today}

\begin{document}

\maketitle

\newpage

\tableofcontents{}

\newpage

\subsection{Carrot Soup (Healthy)}

\paragraph{Ingredients:}
\begin{itemize}
	\item 800g of carrots (Chopped)
	\item 1 Onion
	\item Olive oil
	\item 2 cups of vegetable stock (no salt)
	\item 2 cups of water
	\item 1 teaspoon spoon of minced ginger
	\item 3 strips of orange
\end{itemize}

\paragraph{Directions:}
\begin{itemize}
	\item Put oven on 375F (190C), add \textbf{carrots} and \textbf{onion} into a tray.
	\item Sprinkle \textbf{olive oil}.
	\item Roast for 30-45' and/or until slightly brown.
	\item Put \textbf{onions}, \textbf{carrots}, \textbf{stock}, \textbf{water}, \textbf{ginger} and \textbf{orange} on a saucepan and simmer for 30'
	\item Remove \textbf{orange}
	\item Blend everything
\end{itemize}

\subsection{Tomato Soup (Healthy)}

\paragraph{Ingredients:}
\begin{itemize}
	\item 1.2kg of tomatoes
	\item 1 Onion
	\item Olive oil
	\item 2 cups of vegetable stock (no salt)
	\item 2 cups of water
\end{itemize}

\paragraph{Directions:}
\begin{itemize}
	\item Put oven on 375F (190C) and add \textbf{tomatoes} and \textbf{onion} into a tray.
	\item Sprinkle \textbf{olive oil}.
	\item Roast for 30-45' and/or until slightly brown.
	\item Put \textbf{onions}, \textbf{tomatoes}, \textbf{stock} and \textbf{water} simmer for 90'
	\item Blend everything
\end{itemize}

\subsection{Asparagus Soup (Healthy)}

\paragraph{Ingredients:}
\begin{itemize}
	\item 800g of asparagus
	\item 1 Onion
	\item Olive oil
	\item 2 cups of vegetable stock (no salt)
	\item 2 cups of water
	\item 1 table spoon of yogurt
\end{itemize}

\paragraph{Directions:}
\begin{itemize}
	\item Put oven on 375F (190C) and add \textbf{onion} into a tray.
	\item Sprinkle \textbf{olive oil}.
	\item Roast for 30-45' and/or until slightly brown.
	\item Put \textbf{onions}, \textbf{asparagus} \textbf{stock} and \textbf{water} simmer for 30'
	\item Blend everything
	\item Add \textbf{yogurt}
\end{itemize}

\subsection{Beet Pistachio Bars}

\paragraph{Ingredients (Dry components):}
\begin{itemize}
	\item 1/4 cup brown sugar
	\item 1 cup brown rice flour
	\item 1/4 cup coconut flour
	\item 2 teaspoons baking powder
	\item 2 teaspoon cinnamon
	\item 1/2 teaspoon salt
	\item 1/2 cup dark chocolate chips (optional)
	\item 1/2 cup unsalted shelled pistachios
	\item Zest of 1 lemon
\end{itemize}

\paragraph{Ingredients (Wet components):}
\begin{itemize}
	\item 2 large eggs
	\item 1/3 cup melted coconut oil
	\item 1/2 cup low-fat milk
	\item Some oil for roasting the beets, e.g. sunflower/avocado oil.
	\item 1 pound beets (about 4 medium-sized), peeled and chopped
\end{itemize}

\paragraph{Directions:}
\begin{itemize}
	\item Roast \textbf{beets} on 400F (200C) for 35'. Let them cool.
	\item On a blender add all wet ingredients (Except eggs)
	\item Once blended, move to a container and add eggs.
	\item On a separate container, mix the dry ingredients.
	\item Mix both containers well.
	\item Using a baking pan, add parchment paper and spread evenly. Bake for 30'.
\end{itemize}

\subsection{Vegetable stock}

\paragraph{Ingredients:}
\begin{itemize}
	\item 1 ounce (30g) dried mushrooms*
	\item 4 Tbsp olive oil
	\item 4 cups chopped onion
	\item 2 cups chopped celery
	\item 3 cups chopped carrot
	\item 1 cup chopped fennel bulb (optional)
	\item 2 large garlic cloves, smashed (can leave skins on)
	\item 1 Tbsp tomato paste
	\item 1 Tbsp fresh rosemary
	\item 2 teaspoons dried thyme
	\item 1 teaspoon black peppercorns
	\item 4 bay leaves
	\item 1/2 cup chopped parsley
\end{itemize}

\paragraph{Directions:}
\begin{itemize}
	\item Rehydrate \textbf{dried mushroom}s: Place the dried mushrooms in a large bowl and pour 1 quart (1L) of boiling water over them.
	\item Heat the olive oil over high heat in a large stockpot. Add the chopped \textbf{onions}, \textbf{celery}, \textbf{carrots}, and \textbf{fennel} (if using) and stir to coat. Given that there are so many vegetables, and they have a high moisture content, it may take more heat and longer time to brown than you would expect. Cook until the vegetables begin to brown.
	\item Add \textbf{garlic} and \textbf{tomato paste}. Add the garlic and tomato paste and stir to combine. Cook, stirring often, for 2-3 minutes, or until the tomato paste begins to turn a rusty color.
	\item Add the \textbf{mushrooms} and their soaking water, the \textbf{rosemary}, \textbf{thyme}, \textbf{onion skins} (if using), \textbf{peppercorns},\textbf{ bay leaves}, \textbf{parsley} and 3 additional quarts (3L) of water.
	\item Bring to a simmer and then drop the heat until you just get a bare simmer. The surface of the stock should just barely be bubbling. Cook for 1 1/2 hours.
	\item Strain the stock: Using a spider skimmer or slotted spoon, remove all the big pieces of vegetable and mushroom. Discard or compost.
\end{itemize}

\end{document}